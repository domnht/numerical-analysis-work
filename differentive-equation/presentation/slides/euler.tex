\section{Phương pháp Euler và Euler cải tiến}

\begin{frame}
	\subsection{Phương pháp Euler}
	\frametitle{Phương pháp Euler}

	Xét bài toán Cauchy
	\begin{equation}\label{eqn:eqn615}
		y'=f(x,y),~y(x_0)=y_0
	\end{equation}

	Giả sử hàm $f$ thỏa mãn điều kiện $$|f(x,y_1)-f(x,y_2)|\leqslant L|{y_1}-{y_2}|$$
	và $\left|\frac{\mathrm{d}f}{\mathrm{d}x}\right|=\left|\frac{\partial f}{\partial x}+f\frac{\partial f}{\partial y}\right|\leqslant M$ trong hình chữ nhật:
	$$D=\left\{(x,y)\in\mathbb{R}^2\Big||x-x_0|\leqslant a,|y-y_0|\leqslant b\right\}$$
\end{frame}
\begin{frame}
	\frametitle{Phương pháp Euler}
	Giả sử $x_0$ là giá trị ban đầu và $h$ là số dương cho trước đủ nhỏ với $x_i=x_0+ih$, với $i=0, 1, 2, \ldots, h$ được gọi là độ dài bước.\par
	(\ref{eqn:eqn615}) tương đương với $\mathrm{d}y=f\left(x,y\right)\mathrm{d}x$, lấy tích phân hai vế ta được:
	\begin{center}
		$\int\limits_{y_0}^{y_1}\mathrm{d}y=\int\limits_{x_0}^{x_1}\mathrm{d}x$ hay $y_1=y_0+\int\limits_{x_0}^{x_1} f(x,y)\mathrm{d}x$
	\end{center}
\end{frame}
\begin{frame}
	\frametitle{Phương pháp Euler}
	Giả sử $f(x,y)\approx f\left(x_0,y_0\right)$ với $x_0\leqslant x\leqslant x_1$, khi đó:
	\begin{center}
		$y_1\approx y_0+f\left(x_0,y_0\right)\left(x_1-x_0\right)$ hay $y_1\approx y_0+hf\left(x_0,y_0\right)$
	\end{center}
	Tương tự, với $x_1\leqslant x\leqslant x_2$, ta có $y_2\approx y_1+hf\left(x_1,y_1\right)$.\\
	Từ đó, ta có công thức tổng quát:
	\begin{block}{Phương pháp Euler}
		\begin{equation}\label{eqn:eqn616}
			y_{n+1}=y_n+hf\left(x_n,y_n\right), n=0,~1,~2,~\ldots
		\end{equation}
	\end{block}
	% Công thức (\ref{eqn:eqn616}) được gọi là phương pháp Euler.
\end{frame}
\begin{frame}
	\frametitle{Ví dụ}
	\textbf{Ví dụ 6.7} \textit{(Trang 127)} Bằng phương pháp Euler, giải phương trình gần đúng phương trình:
	$$y'=2xy+\mathrm{e}^{x^2},y(0)=1$$
	trong đoạn $[0;1.5]$, so sánh với nghiệm chính xác $\varphi(x)=(x+1)\mathrm{e}^{x^2}$ của phương trình.\par
\end{frame}
\begin{frame}
	\frametitle{Ví dụ: $y'=2xy+\mathrm{e}^{x^2},y(0)=1$}
	\textbf{Giải}\par
	Với $x_0=0,~y_0=1$, chọn $h=0.25$.\par
	Áp dụng công thức (\ref{eqn:eqn616}), ta có bảng giá trị:
	\begin{table}\begin{tabular}{|c|c|c|c|}\hline
		$n$ & $x_n$  & $y_n$     & $\varphi(x_n)$ \\\hline
		$0$ & $0.0$  & $1.0$     & $1.0$     \\\hline
		$1$ & $0.25$ & $1.25$    & $1.3306$  \\\hline
		$2$ & $0.5$  & $1.6724$  & $1.926$   \\\hline
		$3$ & $0.75$ & $2.4115$  & $3.0713$  \\\hline
		$4$ & $1.0$  & $3.7545$  & $5.4366$  \\\hline
		$5$ & $1.25$ & $6.3114$  & $10.7341$ \\\hline
		$6$ & $1.5$  & $11.4487$ & $23.7193$ \\\hline
	\end{tabular}\end{table}
\end{frame}
\begin{frame}
	\frametitle{Ví dụ: $y'=2xy+\mathrm{e}^{x^2},y(0)=1$}
	Nếu chọn $h=0.1$, ta cũng có bảng giá trị:
	% {\centering\textit{\color{red}Bảng trang 128}\par}
	\begin{table}\begin{tabular}{|c|c|c|c||c|c|c|c|}\hline
		$n$ & $x_n$ & $y_n$    & $\varphi(x_n)$ & $n$  & $x_n$ & $y_n$     & $\varphi(x_n)$ \\\hline
		$0$ & $0.0$  & $1.0$   & $1.0$          & $8$  & $0.8$ & $3.0338$  & $3.4137$       \\\hline
		$1$ & $0.1$ & $1.1$    & $1.1111$       & $9$  & $0.9$ & $3.7088$  & $4.271$        \\\hline
		$2$ & $0.2$ & $1.223$  & $1.249$        & $10$ & $1.0$ & $4.6012$  & $5.4366$       \\\hline
		$3$ & $0.3$ & $1.376$  & $1.4224$       & $11$ & $1.1$ & $5.7933$  & $7.0423$       \\\hline
		$4$ & $0.4$ & $1.568$  & $1.6429$       & $12$ & $1.2$ & $7.4031$  & $9.2855$       \\\hline
		$5$ & $0.5$ & $1.8108$ & $1.926$        & $13$ & $1.3$ & $9.602$   & $12.4648$      \\\hline
		$6$ & $0.6$ & $2.1203$ & $2.2933$       & $14$ & $1.4$ & $12.6404$ & $17.0384$      \\\hline
		$7$ & $0.7$ & $2.518$  & $2.7749$       & $15$ & $1.5$ & $16.8897$ & $23.7193$      \\\hline
	\end{tabular}\end{table}
\end{frame}
\begin{frame}
	\frametitle{Ví dụ}
	\textbf{Ví dụ 6.8} \textit{(Trang 128)} Bằng phương pháp Euler, giải gần đúng phương trình vi phân:
	$$y'+2y=2-{\mathrm{e}^{-4x}},y(0)=1$$
	với $h=0.1$ cho trước.\par
\end{frame}
\begin{frame}\small
	\frametitle{Ví dụ: $y'+2y=2-{\mathrm{e}^{-4x}},y(0)=1$}
	\textbf{Giải}\par
	Phương trình vi phân đã cho có nghiệm chính xác là:
	$$y=1+\frac{1}{2}{\mathrm{e}^{-4x}}-\frac{1}{2}{\mathrm{e}^{-2x}}$$
	Phương trình vi phân đã cho tương đương với: $$y'=2-\mathrm{e}^{-4x}-\frac12\mathrm{e}^{-2x}-2y$$
	Với $x_0=0,~y_0=1$ và $h=0.1$. Khi đó ta có bảng giá trị:
	% {\centering\textit{\color{red}Bảng trang 129}\par}
	\begin{center}\begin{tabular}{|c|c|c|c|}\hline
		$n$ & $x_n$ & $y_n$      & $\varphi(x_n)$ \\\hline
		$0$ & $0.0$ & $1.0$      & $1.0$      \\\hline
		$1$ & $0.1$ & $0.9$      & $0.925795$ \\\hline
		$2$ & $0.2$ & $0.852968$ & $0.889504$ \\\hline
		$3$ & $0.3$ & $0.837441$ & $0.876191$ \\\hline
		$4$ & $0.4$ & $0.839834$ & $0.876284$ \\\hline
		$5$ & $0.5$ & $0.851677$ & $0.883728$ \\\hline
	\end{tabular}\end{center}
\end{frame}
\begin{frame}
	\frametitle{Ví dụ: $y'+2y=2-{\mathrm{e}^{-4x}},y(0)=1$}
	Nếu ta thay đổi độ dài bước lần lượt là $h=0.05,~h=0.01,~h=0.005$ và $h=0.001$, ta có bảng giá trị của nghiệm gần đúng và nghiệm chính xác tại các thời điểm $x=1;x=2;x=3;x=4$ và $x=5$ là:
	% {\centering\textit{\color{red}Bảng trang 129}\par}
	\begin{table}\small
		\begin{tabular}{|c|c|c|c|c|c|}\hline
			\multirow{2}{*}{$x$} & \multirow{2}{*}{$\varphi(x)$} & \multicolumn{4}{c|}{Nghiệm gần đúng}\\\cline{3-6}
			&&$h=0.05$&$h=0.01$&$h=0.005$&$h=0.001$\\\hline
			$1$&$0.9414902$&$0.9364698$&$0.9404994$&$0.9409957$&$0.9413914$\\\hline
			$2$&$0.9910099$&$0.9911126$&$0.9910193$&$0.9910139$&$0.9910106$\\\hline
			$3$&$0.9987637$&$0.9988982$&$0.9987890$&$0.9987763$&$0.9987662$\\\hline
			$4$&$0.9998323$&$0.9998657$&$0.9998390$&$0.9998357$&$0.9998330$\\\hline
			$5$&$0.9999773$&$0.9999837$&$0.9999786$&$0.9999780$&$0.9999774$\\\hline
		\end{tabular}
	\end{table}
\end{frame}
\begin{frame}
	\frametitle{Sai số phương pháp}
	Phần trăm sai số giữa nghiệm chính xác và nghiệm gần đúng được xác định bởi:\par
	\begin{equation}\label{eqn:eqn617}
		P=\frac{|n_\text{\tiny chính xác}-n_\text{\tiny gần đúng}|}{n_\text{\tiny chính xác}}\times 100\%
	\end{equation}

	Áp dụng (\ref{eqn:eqn617}), ta có bảng tính toán phần trăm sai số giữa nghiệm chính xãc và nghiệm gần đúng:
	% {\centering\textit{\color{red}Bảng trang 130}\par}
	\begin{table}\begin{tabular}{|c|c|c|c|c|}\hline
		$x$&$h=0.05$&$h=0.01$&$h=0.005$&$h=0.001$\\\hline
		$1$ & $0.53\%$ & $0.105\%$ & $0.053\%$ & $0.0105\%$ \\\hline
		$2$ & $0.01\%$ & $0.0094\%$ & $0.00041\%$ & $0.00007\%$ \\\hline
		$3$ & $0.013\%$ & $0.0025\%$ & $0.0013\%$ & $0.00025\%$ \\\hline
		$4$ & $0.0033\%$ & $0.00067\%$ & $0.00034\%$ & $0.000067\%$ \\\hline
		$5$ & $0.00064\%$ & $0.00013\%$ & $0.000068\%$ & $0.000014\%$ \\\hline
	\end{tabular}\end{table}
\end{frame}
\begin{frame}
	\frametitle{Nhận xét}
	% \textbf{Nhận xét 6.1}\par
	\begin{itemize}
		\item Nếu độ dài bước $h$ càng nhỏ, sai số giữa nghiệm gần đúng và nghiệm chính xác càng nhỏ.
		\item Nói chung sai số giữa nghiệm gần đúng và nghiệm chính xác sẽ tăng nếu giá trị $x$ tăng.
	\end{itemize}
\end{frame}

\begin{frame}
	\subsection{Phương pháp Euler cải tiến}
	\frametitle{Phương pháp Euler cải tiến}
	Trong phương pháp Euler cải tiến (thứ nhất), giá trị tiếp theo của nghiệm $y_{n+1}$ được tính thông qua việc tính toán các giá trị trung gian $x_{n+\frac12}$, $y_{n+\frac12}$ và $f_{n+\frac12}$:\par
	\begin{block}{Phương pháp Euler cải tiến thứ nhất}
	\begin{equation}\label{eqn:eqn618}\begin{cases}
		x_{n+\frac12}&=x_n+\frac h2\\
		y_{n+\frac12}&=y_n+\frac h2f_n\\
		f_{n+\frac12}&=f\left(x_n+\frac h2, y_n+\frac h2f_n\right)\\
		y_{n+1}&=y_n+hf_{n+\frac12}
	\end{cases}\end{equation}
	\end{block}
\end{frame}
\begin{frame}
	\frametitle{Ví dụ}
	\textbf{Ví dụ 6.9} \textit{(Trang 130)} Áp dụng phương pháp Euler cải tiến thứ nhất, giải phương trình vi phân sau:
	$$y'=\frac{2y}{x}+x,y(1)=1$$
	trên đoạn $[1;1.4]$ và độ dài bước $h=0.1$.\par
\end{frame}
\begin{frame}
	\frametitle{Ví dụ: $y'=\frac{2y}{x}+x,y(1)=1$}
	\textbf{Giải}\par
	Phương trình đã cho có nghiệm gần đúng là $\varphi(x)=x^2+x^2\ln x$.\par
	Đặt $f(x,y)=\frac{2y}{x}+x$, ta có bảng giá trị:\par
	% {\centering\textit{\color{red}Bảng trang 131}\par}
	\begin{table}\begin{tabular}{|c|c|c|c|}\hline
		$n$ & $x_n$ & $y_n$      & $\varphi(x_n)$ \\ \hline
		$0$ & $1.0$ & $1.0$      & $1.0$          \\ \hline
		$1$ & $1.1$ & $1.324048$ & $1.325325$     \\ \hline
		$2$ & $1.2$ & $1.699816$ & $1.702543$     \\ \hline
		$3$ & $1.3$ & $2.12905$  & $2.133396$     \\ \hline
		$4$ & $1.4$ & $2.613357$ & $2.619486$     \\ \hline
	\end{tabular}\end{table}
\end{frame}
