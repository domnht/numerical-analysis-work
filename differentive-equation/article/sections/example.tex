\textbf{Câu 1.} Bằng các phương pháp Euler, Euler cải tiến, Runge – Kutta 4, Adams, tìm nghiệm gần đúng của phương trình $$y'=y-x^2+1$$ trên đoạn $[0;1]$ với $h=0.1$, biết $y(0)=0.5$. So sánh với nghiệm chính xác $\varphi(x)=(x+1)^2-0.5\mathrm{e}^x$.\par
\textbf{Giải}\par
$\star$~Phương pháp Euler, Euler cải tiến 1 (Euler 1) và Euler cải tiến 2 (Euler 2):
\begin{longtable}{|c|c|c|c|c|c|c|c|c|}\hline
	$n$  & $x_n$ & $y_n$ (Euler) & Sai số & $y_n$ (Euler 1) & Sai số & $y_n$ (Euler 2) & Sai số & $\varphi(x_n)$ \\\hline
	\endhead
	$0$  & $0.0$ & $0.5$         & $0.0\%$  & $0.5$           & $0.0\%$  & $0.5$           & $0.0\%$  & $0.5$          \\\hline
	$1$  & $0.1$ & $0.65$        & $1.13\%$ & $0.65725$       & $0.03\%$ & $0.657$         & $0.06\%$ & $0.657415$     \\\hline
	$2$  & $0.2$ & $0.814$       & $1.84\%$ & $0.828961$      & $0.04\%$ & $0.82805$       & $0.15\%$ & $0.829299$     \\\hline
	$3$  & $0.3$ & $0.9914$      & $2.33\%$ & $1.014552$      & $0.05\%$ & $1.012523$      & $0.25\%$ & $1.015071$     \\\hline
	$4$  & $0.4$ & $1.18154$     & $2.68\%$ & $1.21338$       & $0.06\%$ & $1.209726$      & $0.36\%$ & $1.214088$     \\\hline
	$5$  & $0.5$ & $1.383694$    & $2.94\%$ & $1.424735$      & $0.06\%$ & $1.418897$      & $0.47\%$ & $1.425639$     \\\hline
	$6$  & $0.6$ & $1.597063$    & $3.15\%$ & $1.647832$      & $0.07\%$ & $1.639195$      & $0.59\%$ & $1.648941$     \\\hline
	$7$  & $0.7$ & $1.82077$     & $3.31\%$ & $1.881805$      & $0.07\%$ & $1.869693$      & $0.71\%$ & $1.883124$     \\\hline
	$8$  & $0.8$ & $2.053847$    & $3.45\%$ & $2.125694$      & $0.07\%$ & $2.10937$       & $0.84\%$ & $2.12723$      \\\hline
	$9$  & $0.9$ & $2.295231$    & $3.57\%$ & $2.378442$      & $0.07\%$ & $2.3571$        & $0.97\%$ & $2.380198$     \\\hline
	$10$ & $1.0$ & $2.543755$    & $3.68\%$ & $2.638878$      & $0.08\%$ & $2.611643$      & $1.11\%$ & $2.640859$     \\\hline
\end{longtable}

$\star$~Phương pháp Runge – Kutta 4:
\begin{longtable}{|c|c|c|c|c|c|}\hline
	$n$                & $x_n$ & $y_n$  & $hf(x_n, y_n)$ & $\varphi(x_n)$ & Sai số \\ \hline
	\endhead
	\multirow{4}{*}{0} & 0.0   & 0.5    & 0.15           &                &          \\ \cline{2-4}
	                   & 0.05  & 0.575  & 0.1573         &                &          \\ \cline{2-4}
	                   & 0.05  & 0.5786 & 0.1576         &                &          \\ \cline{2-4}
	                   & 0.1   & 0.6576 & 0.1648         &                &          \\ \hline
	\multicolumn{4}{|c|}{$y_1 = 0.657414375$}             & 0.657414541   & 0.00002525\% \\ \hline
	\multirow{4}{*}{1} & 0.1   & 0.6574 & 0.1647         &                &          \\ \cline{2-4}
	                   & 0.15  & 0.7398 & 0.1717         &                &          \\ \cline{2-4}
	                   & 0.15  & 0.7433 & 0.1721         &                &          \\ \cline{2-4}
	                   & 0.2   & 0.8295 & 0.1789         &                &          \\ \hline
	\multicolumn{4}{|c|}{$y_2 = 0.829298276$}             & 0.829298621   & 0.0000416\% \\ \hline
	\multirow{4}{*}{2} & 0.2   & 0.8293 & 0.1789         &                &          \\ \cline{2-4}
	                   & 0.25  & 0.9188 & 0.1856         &                &          \\ \cline{2-4}
	                   & 0.25  & 0.9221 & 0.186          &                &          \\ \cline{2-4}
	                   & 0.3   & 1.0153 & 0.1925         &                &          \\ \hline
	\multicolumn{4}{|c|}{$y_3 = 1.015070058$}            & 1.015070596    & 0.000053\% \\ \hline
	\multirow{4}{*}{3} & 0.3   & 1.0151 & 0.1925         &                &          \\ \cline{2-4}
	                   & 0.35  & 1.1113 & 0.1989         &                &          \\ \cline{2-4}
	                   & 0.35  & 1.1145 & 0.1992         &                &          \\ \cline{2-4}
	                   & 0.4   & 1.2143 & 0.2054         &                &          \\ \hline
	\multicolumn{4}{|c|}{$y_4 = 1.214086906$}            & 1.214087651    & 0.00006136\% \\ \hline
	\multirow{4}{*}{4} & 0.4   & 1.2141 & 0.2054         &                &          \\ \cline{2-4}
	                   & 0.45  & 1.3168 & 0.2114         &                &          \\ \cline{2-4}
	                   & 0.45  & 1.3198 & 0.2117         &                &          \\ \cline{2-4}
	                   & 0.5   & 1.4258 & 0.2176         &                &          \\ \hline
	\multicolumn{4}{|c|}{$y_5 = 1.425638396$}            & 1.425639365    & 0.00006797\% \\ \hline
	\multirow{4}{*}{5} & 0.5   & 1.4256 & 0.2176         &                &          \\ \cline{2-4}
	                   & 0.55  & 1.5344 & 0.2232         &                &          \\ \cline{2-4}
	                   & 0.55  & 1.5372 & 0.2235         &                &          \\ \cline{2-4}
	                   & 0.6   & 1.6491 & 0.2289         &                &          \\ \hline
	\multicolumn{4}{|c|}{$y_6 = 1.648939390$}            & 1.6489406      & 0.00007338\% \\ \hline
	\multirow{4}{*}{6} & 0.6   & 1.6489 & 0.2289         &                &          \\ \cline{2-4}
	                   & 0.65  & 1.7634 & 0.2341         &                &          \\ \cline{2-4}
	                   & 0.65  & 1.766  & 0.2343         &                &          \\ \cline{2-4}
	                   & 0.7   & 1.8833 & 0.2393         &                &          \\ \hline
	\multicolumn{4}{|c|}{$y_7 = 1.883122179$}            & 1.883123646    & 0.0000779\% \\ \hline
	\multirow{4}{*}{7} & 0.7   & 1.8831 & 0.2393         &                &          \\ \cline{2-4}
	                   & 0.75  & 2.0028 & 0.244          &                &          \\ \cline{2-4}
	                   & 0.75  & 2.0051 & 0.2443         &                &          \\ \cline{2-4}
	                   & 0.8   & 2.1274 & 0.2487         &                &          \\ \hline
	\multicolumn{4}{|c|}{$y_8 = 2.127227791$}            & 2.127229536    & 0.00008203\% \\ \hline
	\multirow{4}{*}{8} & 0.8   & 2.1272 & 0.2487         &                &          \\ \cline{2-4}
	                   & 0.85  & 2.2516 & 0.2529         &                &          \\ \cline{2-4}
	                   & 0.85  & 2.2537 & 0.2531         &                &          \\ \cline{2-4}
	                   & 0.9   & 2.3803 & 0.257          &                &          \\ \hline
	\multicolumn{4}{|c|}{$y_9 = 2.380196402$}            & 2.380198444    & 0.00008579\% \\ \hline
	\multirow{4}{*}{9} & 0.9   & 2.3802 & 0.257          &                &          \\ \cline{2-4}
	                   & 0.95  & 2.5087 & 0.2606         &                &          \\ \cline{2-4}
	                   & 0.95  & 2.5105 & 0.2608         &                &          \\ \cline{2-4}
	                   & 1.0   & 2.641  & 0.2641         &                &          \\ \hline
	\multicolumn{4}{|c|}{$y_10 = 2.64085672$}            & 2.640859086    & 0.00008959\% \\ \hline
\end{longtable}

$\star$~Phương pháp Adams:
\begin{longtable}{|c|c|c|c|c|c|c|c|}\hline
	i  & $x_i$ & Nội suy   & Sai số       & Ngoại suy   & Sai số       & RK4      & $\varphi(x)$ \\ \hline
	\endhead
	4  & 0.4   & 1.214087  & 0.00000079\% & 1.214089    & 0.00000096\% & 1.214087 & 1.214088     \\ \hline
	5  & 0.5   & 1.425639  & 0.00000034\% & 1.425641    & 0.00000132\% & 1.425638 & 1.425639     \\ \hline
	6  & 0.6   & 1.648940  & 0.0000009\%  & 1.648942    & 0.00000067\% & 1.648939 & 1.648941     \\ \hline
	7  & 0.7   & 1.883122  & 0.00000091\% & 1.883125    & 0.00000062\% & 1.883122 & 1.883124     \\ \hline
	8  & 0.8   & 2.127228  & 0.00000099\% & 2.127231    & 0.0000005\%  & 2.127228 & 2.127230     \\ \hline
	9  & 0.9   & 2.380196  & 0.00000063\% & 2.380200    & 0.00000084\% & 2.380196 & 2.380198     \\ \hline
	10 & 1.0   & 2.640857  & 0.00000083\% & 2.640861    & 0.00000064\% & 2.640857 & 2.640859     \\ \hline
\end{longtable}

\textbf{Câu 2.} Bằng các phương pháp Euler, Euler cải tiến, Runge – Kutta 4, Adams, tìm nghiệm gần đúng của phương trình $$y'=y-\frac{2x}{y}$$ trên đoạn $[0;1]$, với $y(0)=1$ và $h=0.2$.\par
% So sánh với nghiệm chính xác $\varphi(x)=\sqrt{2x+1}$.\par
\textbf{Giải}\par

$\star$~Phương pháp Euler:
\begin{longtable}{|c|c|c|c|c|}\hline
	$n$ & $x_n$ & $y_n$      & $y_n$ (Euler 1) & $y_n$ (Euler 2) \\ \hline
	\endhead
	0   & 0.0   & 1.0        & 1.0             & 1.0             \\ \hline
	1   & 0.2   & 1.2        & 1.18363636      & 1.18666667      \\ \hline
	2   & 0.4   & 1.37333333 & 1.34265567      & 1.35070637      \\ \hline
	3   & 0.6   & 1.53149515 & 1.48501361      & 1.50134346      \\ \hline
	4   & 0.8   & 1.68108457 & 1.61522499      & 1.64448119      \\ \hline
	5   & 1.0   & 1.82694818 & 1.73618226      & 1.78485683      \\ \hline
\end{longtable}

$\star$~Phương pháp Runge – Kutta 4:
\begin{longtable}{|c|c|c|c|}\hline
	$n$                & $x_n$ & $y_n$  & $hf(x_n, y_n)$ \\ \hline
	\endhead
	\multirow{4}{*}{0} & 0     & 1      & 0.2        \\ \cline{2-4} 
	                   & 0.1   & 1.1    & 0.1836     \\ \cline{2-4} 
	                   & 0.1   & 1.0918 & 0.1817     \\ \cline{2-4} 
	                   & 0.2   & 1.1817 & 0.1686     \\ \hline
	\multicolumn{4}{|c|}{$y_1 = 1.183229287445307$}    \\ \hline
	\multirow{4}{*}{1} & 0.2   & 1.1832 & 0.169      \\ \cline{2-4} 
	                   & 0.3   & 1.2677 & 0.1589     \\ \cline{2-4} 
	                   & 0.3   & 1.2627 & 0.1575     \\ \cline{2-4} 
	                   & 0.4   & 1.3407 & 0.1488     \\ \hline
	\multicolumn{4}{|c|}{$y_2 = 1.3416669298526065$}   \\ \hline
	\multirow{4}{*}{2} & 0.4   & 1.3417 & 0.1491     \\ \cline{2-4} 
	                   & 0.5   & 1.4162 & 0.142      \\ \cline{2-4} 
	                   & 0.5   & 1.4127 & 0.141      \\ \cline{2-4} 
	                   & 0.6   & 1.4826 & 0.1347     \\ \hline
	\multicolumn{4}{|c|}{$y_3 = 1.4832814583502616$}   \\ \hline
	\multirow{4}{*}{3} & 0.6   & 1.4833 & 0.1349     \\ \cline{2-4} 
	                   & 0.7   & 1.5507 & 0.1296     \\ \cline{2-4} 
	                   & 0.7   & 1.5481 & 0.1287     \\ \cline{2-4} 
	                   & 0.8   & 1.612  & 0.1239     \\ \hline
	\multicolumn{4}{|c|}{$y_4 = 1.6125140416775265$}   \\ \hline
	\multirow{4}{*}{4} & 0.8   & 1.6125 & 0.1241     \\ \cline{2-4} 
	                   & 0.9   & 1.6745 & 0.1199     \\ \cline{2-4} 
	                   & 0.9   & 1.6725 & 0.1192     \\ \cline{2-4} 
	                   & 1     & 1.7318 & 0.1154     \\ \hline
	\multicolumn{4}{|c|}{$y_5 = 1.7321418826911932$}   \\ \hline
\end{longtable}

$\star$~Phương pháp Adams
\begin{longtable}{|c|c|c|c|c|}\hline
	$i$ & $x_i$ & Nội suy     & Ngoại suy   & RK4         \\ \hline
	\endhead
	4   & 0.8   & 1.612467    & 1.611428559 & 1.612514042 \\ \hline
	5   & 1     & 1.732124625 & 1.731735132 & 1.732141883 \\ \hline
\end{longtable}

\textbf{Câu 3.} Tốc độ tăng kích thước của một khối u trong cơ thể được mô tả bởi phương trình: $$\frac{\text{d}V}{\text{d}t}=\frac{2}{3}\left(\ln{2,3} - \ln{V} \right)$$ với $t$ là thời gian đo bằng ngày. Giả sử một bệnh nhân có kích thước khối u ban đầu là $1$mm\textsuperscript{3}. Tính kích thước khối u đo sau $1$ tuần, với độ dài bước là $h=0,5$.\par
\textbf{Giải}\par
Theo giả thiết ta có: ${V_t}'= \frac{2}{3}\left(\ln{2,3}-\ln{V} \right)$, $V(t_0)=1$mm\textsuperscript{3}. Áp dụng công thức RK4 với độ dài bước $h=0,5$ ta có được bảng giá trị sau:


% \begin{center}
% \Pisymbol{dingbat}{70} \Pisymbol{dingbat}{69}\\
% \Pisymbol{dingbat}{71} \Pisymbol{dingbat}{72}
% \end{center}
 