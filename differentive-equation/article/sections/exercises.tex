% % \section{Bài tập}

% \textbf{Câu 4.} Tìm nghiệm gần đúng của các phương trình sau bằng phương pháp Euler:
% \begin{enumerate}[label=\alph*)]
% 	\item $y'=1-y$, $y(0)=0$, $h=0.1$ trên đoạn $[0;0.3]$.
% 	\item $y'=\frac{y-x}{1+x}$, $y(0)=1$, $h=0.02$ trên đoạn $[0;0.1]$.
% 	\item $y'=3x+\frac12$, $y(0)=1$, $h=0.05$ trên đoạn $[0;0.2]$.
% 	\item $y'=x+y+xy$, $y(0)=1$, $h=0.02$ trên đoạn $[0;0.1]$.
% 	\item $y'=1+\ln(x+y)$, $y(0)=1$, $h=0.1$ trên đoạn $[0;0.2]$.
% 	\item $y'=(y+x)^2$, $y(0)=1$, $h=0.1$ trên đoạn $[0;1]$.
% 	\item $y'=-5x^4y^2$, $y(0)=1$, $h=0.2$ trên đoạn $[0;1]$.
% \end{enumerate}

% \textbf{Giải}\par

% c) $x_0=0$, $y_0=1$, $h=0.05$, ta có bảng giá trị:
% \begin{longtable}{|c|c|c|c|c|}\hline
% $n$ & $x_n$ & $y_n$ & $y_n$ (Euler cải tiến 1) & $y_n$(Euler cải tiến 2) \\ \hline
% $0$ & $0.0$ & $1.0$ & $1.0$ & $1.0$ \\ \hline
% $1$ & $0.05$ & $1.05$ & $1.051281$ & $1.051313$ \\ \hline
% $2$ & $0.1$ & $1.102625$ & $1.105444$ & $1.105473$ \\ \hline
% $3$ & $0.15$ & $1.158256$ & $1.162892$ & $1.162879$ \\ \hline
% $4$ & $0.2$ & $1.217294$ & $1.224049$ & $1.223946$ \\ \hline
% \end{longtable}
% d) $x_0=0$, $y_0=0$, $h=0.2$, ta có bảng giá trị:

% \begin{longtable}{|c|c|c|c|c|}\hline
% 	$n$ & $x_n$ & $y_n$ & $y_n$ (Euler cải tiến 1) & $y_n$(Euler cải tiến 2) \\ \hline
% 	\endhead
% 	$0$ & $0.0$ & $0.0$ & $0.0$ & $0.0$ \\ \hline
% 	$1$ & $0.1$ & $0.1$ & $0.095$ & $0.095$ \\ \hline
% 	$2$ & $0.2$ & $0.19$ & $0.180975$ & $0.18075$ \\ \hline
% 	$3$ & $0.3$ & $0.271$ & $0.258782$ & $0.258163$ \\ \hline
% \end{longtable}

% b) $x_0=0$, $y_0=1$, $h=0.02$, ta có bảng giá trị:
% \begin{longtable}{|c|c|c|c|c|}\hline
% 	$n$ & $x_n$ & $y_n$ & $y_n$ (Euler cải tiến 1) & $y_n$(Euler cải tiến 2) \\ \hline
% 	\endhead
% 	$0$ & $0.0$ & $1.0$ & $1.0$ & $1.0$ \\ \hline
% 	$1$ & $0.02$ & $1.02$ & $1.019802$ & $1.019804$ \\ \hline
% 	$2$ & $0.04$ & $1.039608$ & $1.039212$ & $1.039218$ \\ \hline
% 	$3$ & $0.06$ & $1.058831$ & $1.058237$ & $1.058248$ \\ \hline
% 	$4$ & $0.08$ & $1.077677$ & $1.076885$ & $1.076904$ \\ \hline
% 	$5$ & $0.1$ & $1.096152$ & $1.095162$ & $1.09519$ \\ \hline
% \end{longtable}

% c) $x_0=0$, $y_0=1$, $h=0.05$, ta có bảng giá trị:
% \begin{longtable}{|c|c|c|c|c|}\hline
% 	$n$ & $x_n$ & $y_n$ & $y_n$ (Euler cải tiến 1) & $y_n$(Euler cải tiến 2) \\ \hline
% 	\endhead
% 	$0$ & $0.0$ & $1.0$ & $1.0$ & $1.0$ \\ \hline
% 	$1$ & $0.05$ & $1.025$ & $1.02875$ & $1.02875$ \\ \hline
% 	$2$ & $0.1$ & $1.0575$ & $1.065$ & $1.065$ \\ \hline
% 	$3$ & $0.15$ & $1.0975$ & $1.10875$ & $1.10875$ \\ \hline
% 	$4$ & $0.2$ & $1.145$ & $1.16$ & $1.16$ \\ \hline
% \end{longtable}

% d) $x_0=0$, $y_0=1$, $h=0.02$, ta có bảng giá trị:
% \begin{longtable}{|c|c|c|c|c|}\hline
% 	$n$ & $x_n$ & $y_n$ & $y_n$ (Euler cải tiến 1) & $y_n$(Euler cải tiến 2) \\ \hline
% 	\endhead
% 	$0$ & $0.0$ & $1.0$ & $1.0$ & $1.0$ \\ \hline
% 	$1$ & $0.02$ & $1.02$ & $1.020602$ & $1.020604$ \\ \hline
% 	$2$ & $0.04$ & $1.041208$ & $1.042445$ & $1.042443$ \\ \hline
% 	$3$ & $0.06$ & $1.063665$ & $1.065572$ & $1.065559$ \\ \hline
% 	$4$ & $0.08$ & $1.087415$ & $1.09003$ & $1.089998$ \\ \hline
% 	$5$ & $0.1$ & $1.112503$ & $1.115867$ & $1.115807$ \\ \hline
% \end{longtable}

% e) $x_0=0$, $y_0=1$, $h=0.1$, ta có bảng giá trị:
% \begin{longtable}{|c|c|c|c|c|}\hline
% 	$n$ & $x_n$ & $y_n$ & $y_n$ (Euler cải tiến 1) & $y_n$(Euler cải tiến 2) \\ \hline
% 	\endhead
% 	$0$ & $0.0$ & $1.0$ & $1.0$ & $1.0$ \\ \hline
% 	$1$ & $0.1$ & $1.1$ & $1.109531$ & $1.109116$ \\ \hline
% 	$2$ & $0.2$ & $1.218232$ & $1.237222$ & $1.236081$ \\ \hline
% \end{longtable}

% f) $x_0=0$, $y_0=1$, $h=0.1$ ta có bảng giá trị:
% \begin{longtable}{|c|c|c|c|c|}\hline
% 	$n$ & $x_n$ & $y_n$ & $y_n$ (Euler cải tiến 1) & $y_n$(Euler cải tiến 2) \\ \hline
% 	\endhead
% 	$0$ & $0.0$ & $1.0$ & $1.0$ & $1.0$ \\ \hline
% 	$1$ & $0.1$ & $1.1$ & $1.121$ & $1.122$ \\ \hline
% 	$2$ & $0.2$ & $1.244$ & $1.302048$ & $1.300921$ \\ \hline
% 	$3$ & $0.3$ & $1.452514$ & $1.579223$ & $1.567124$ \\ \hline
% 	$4$ & $0.4$ & $1.759644$ & $2.022661$ & $1.974635$ \\ \hline
% 	$5$ & $0.5$ & $2.22605$ & $2.787806$ & $2.628147$ \\ \hline
% 	$6$ & $0.6$ & $2.969185$ & $4.291918$ & $3.754367$ \\ \hline
% 	$7$ & $0.7$ & $4.243093$ & $8.059989$ & $5.924101$ \\ \hline
% 	$8$ & $0.8$ & $6.686511$ & $24.054293$ & $10.920429$ \\ \hline
% 	$9$ & $0.9$ & $12.291295$ & $335.318823$ & $26.489364$ \\ \hline
% 	$10$ & $1.0$ & $29.692322$ & $3586457.107023$ & $111.099159$ \\ \hline
% \end{longtable}

% g) $x_0=0$, $y_0=1$, $h=0.2$ ta có bảng giá trị:
% \begin{longtable}{|c|c|c|c|c|}\hline
% 	$n$ & $x_n$ & $y_n$ & $y_n$ (Euler cải tiến 1) & $y_n$(Euler cải tiến 2) \\ \hline
% 	$0$ & $0.0$ & $1.0$ & $1.0$ & $1.0$ \\ \hline
% 	$1$ & $0.2$ & $1.0$ & $0.9999$ & $0.9992$ \\ \hline
% 	$2$ & $0.4$ & $0.9984$ & $0.991815$ & $0.985642$ \\ \hline
% 	$3$ & $0.6$ & $0.972882$ & $0.931885$ & $0.911874$ \\ \hline
% 	$4$ & $0.8$ & $0.850216$ & $0.747801$ & $0.709949$ \\ \hline
% 	$5$ & $1.0$ & $0.554129$ & $0.48468$ & $0.453194$ \\ \hline
% \end{longtable}

% \textbf{Câu 5.} Tìm nghiệm gần đúng của các phương trình sau bằng phương pháp Euler cải tiến:
% \begin{enumerate}[label=\alph*)]
% 	\item $y'=\frac{y-x}{1+x}$, $y(0)=1$, $h=0.02$ trên đoạn $[0;0.1]$.
% 	\item $y'=3x+\frac12y$, $y(0)=1$, $h=0.05$ trên đoạn $[0;0.2]$.
% 	\item $y'=x^2+y$, $y(0)=1$, $h=0.05$ trên đoạn $[0;0.2]$.
% 	\item $y'=1+y^2$, $y(0)=0$, $h=0.2$ trên đoạn $[0;0.6]$.
% \end{enumerate}

% \textbf{Giải}\par
% a) (Câu 4b)\par
% b) (Câu 4c)\par
% c) $x_0=0$, $y_0=1$, $h=0.05$, ta có bảng giá trị:
% \begin{longtable}{|c|c|c|c|c|}\hline
% 	$n$ & $x_n$ & $y_n$ & $y_n$ (Euler cải tiến 1) & $y_n$(Euler cải tiến 2) \\ \hline
% 	$0$ & $0.0$ & $1.0$ & $1.0$ & $1.0$ \\ \hline
% 	$1$ & $0.05$ & $1.05$ & $1.051281$ & $1.051313$ \\ \hline
% 	$2$ & $0.1$ & $1.102625$ & $1.105444$ & $1.105473$ \\ \hline
% 	$3$ & $0.15$ & $1.158256$ & $1.162892$ & $1.162879$ \\ \hline
% 	$4$ & $0.2$ & $1.217294$ & $1.224049$ & $1.223946$ \\ \hline
% \end{longtable}

% d) $x_0=0$, $y_0=0$, $h=0.2$, ta có bảng giá trị:
% \begin{longtable}{|c|c|c|c|c|}\hline
% $n$ & $x_n$ & $y_n$ & $y_n$ (Euler cải tiến 1) & $y_n$(Euler cải tiến 2) \\ \hline
% 	$0$ & $0.0$ & $0.0$ & $0.0$ & $0.0$ \\ \hline
% 	$1$ & $0.2$ & $0.2$ & $0.202$ & $0.204$ \\ \hline
% 	$2$ & $0.4$ & $0.408$ & $0.420737$ & $0.424808$ \\ \hline
% 	$3$ & $0.6$ & $0.641293$ & $0.67872$ & $0.68398$ \\ \hline
% \end{longtable}

% \textbf{Câu 6.} Tìm nghiệm gần đúng của các phương trình sau bằng phương pháp RK4:
% \begin{enumerate}[label=\alph*)]
% 	\item $y'=2+\sqrt{xy}$, $y(1)=1$, $h=0.2$ trên đoạn $[0;2]$
% 	\item $y'=\frac{y}{x}-y^2$, $y(1)=1$, $h=0.2$ trên đoạn $[1;2]$
% 	\item $y'=x-\sin y$, $y(0)=0$, $h=0.1$ trên đoạn $\left[0;\frac{\pi}{2}\right]$
% 	\item $y'=x^2+y^2$, $y(0)=1$, $h=0.1$ trên đoạn $[0;1]$
% 	\item $y'=x-\sqrt{y}$, $y(0)=1$, $h=0.2$ trên đoạn $[0;1]$
% \end{enumerate}

% \textbf{Giải}\par

% a) $y'=2+\sqrt{xy}$, $y(1)=1$, $h=0.2$.\par
% Ta có sơ đồ tính toán như sau:

% \begin{longtable}{|c|c|c|c|} \hline
% $n$ & $x_o$ &$y_o$ &$hf(x_n,y_n)$ \\ \hline
% \endhead

% \multirow{4}{*}{$0$}
% & $1.0$ & $1.0$ & $0.6$ \\ \cline{2-4}
% & $1.1$ & $1.3$ & $0.6392$ \\ \cline{2-4}
% & $1.1$ & $1.3196$ & $0.6410$ \\ \cline{2-4}
% & $1.2$ & $1.6410$ & $0.6807$ \\ \hline
% \multicolumn{4}{|c|}{$y_1 = 1.640150501121759$} \\ \hline

% \multirow{4}{*}{$1$}
% & $1.2$ & $1.6402$ & $0.6806$ \\ \cline{2-4}
% & $1.3$ & $1.9804$ & $0.7209$ \\ \cline{2-4}
% & $1.3$ & $2.0006$ & $0.7225$ \\ \cline{2-4}
% & $1.4$ & $2.3627$ & $0.7637$ \\ \hline
% \multicolumn{4}{|c|}{$y_2 = 2.3620215847729593$} \\ \hline

% \multirow{4}{*}{$2$}
% & $1.4$ & $2.3620$ & $0.7637$ \\ \cline{2-4}
% & $1.5$ & $2.7439$ & $0.8057$ \\ \cline{2-4}
% & $1.5$ & $2.7649$ & $0.8073$ \\ \cline{2-4}
% & $1.6$ & $3.1693$ & $0.8504$ \\ \hline
% \multicolumn{4}{|c|}{$y_3 = 3.168716003302697$} \\ \hline

% \multirow{4}{*}{$3$}
% & $1.6$ & $3.1687$ & $0.8503$ \\ \cline{2-4}
% & $1.7$ & $3.5939$ & $0.8944$ \\ \cline{2-4}
% & $1.7$ & $3.6159$ & $0.8959$ \\ \cline{2-4}
% & $1.8$ & $4.0646$ & $0.9410$ \\ \hline
% \multicolumn{4}{|c|}{$y_4 = 4.064004846121863$} \\ \hline

% \multirow{4}{*}{$4$}
% & $1.8$ & $4.0640$ & $0.9409$ \\ \cline{2-4}
% & $1.9$ & $4.5345$ & $0.9870$ \\ \cline{2-4}
% & $1.9$ & $4.5575$ & $0.9885$ \\ \cline{2-4}
% & $2.0$ & $5.0525$ & $1.0358$ \\ \hline
% \multicolumn{4}{|c|}{$y_5 = 5.051980957085838$} \\ \hline

% \end{longtable}

% Vậy nghiệm gần đúng của phương trình là $y_5 = 5.051980957085838$.\par

% b) $y'=\frac{y}{x}-y^2$, $y(1)=1$, $h=0.2$ trên đoạn $[1;2]$\par
% Ta lập được sơ đồ tính toán như sau:

% \begin{longtable}{|c|c|c|c|}\hline
% $n$&$x_o$ &$y_o$ &$hf(x_n,y_n)$ \\ \hline
% \endhead

% \multirow{4}{*}{$0$}
% &1.0 &1.0 &0.0000 \\ \cline{2-4}
% &1.1 &1.0 &-0.0182 \\ \cline{2-4}
% &1.1 &0.9909 &-0.0162 \\ \cline{2-4}
% &1.2 &0.9838 &-0.0296 \\ \hline
% \multicolumn{4}{|c|}{$y_1 = 0.9836006941450569$} \\ \hline

% \multirow{4}{*}{$1$}
% &1.2 &0.9836 &-0.0296 \\ \cline{2-4}
% &1.3 &0.9688 &-0.0387 \\ \cline{2-4}
% &1.3 &0.9643 &-0.0376 \\ \cline{2-4}
% &1.4 &0.9460 &-0.0438 \\ \hline
% \multicolumn{4}{|c|}{$y_2 = 0.9459390018738288$} \\ \hline

% \multirow{4}{*}{$2$}
% &1.4 &0.9459 &-0.0438 \\ \cline{2-4}
% &1.5 &0.9240 &-0.0476 \\ \cline{2-4}
% &1.5 &0.9222 &-0.0471 \\ \cline{2-4}
% &1.6 &0.8988 &-0.0492 \\ \hline
% \multicolumn{4}{|c|}{$y_3 = 0.8988702115243877$} \\ \hline

% \multirow{4}{*}{$3$}
% &1.6 &0.8989 &-0.0492 \\ \cline{2-4}
% &1.7 &0.8743 &-0.0500 \\ \cline{2-4}
% &1.7 &0.8739 &-0.0499 \\ \cline{2-4}
% &1.8 &0.8489 &-0.0498 \\ \hline
% \multicolumn{4}{|c|}{$y_4 = 0.849051617841853$} \\ \hline

% \multirow{4}{*}{$4$}
% &1.8 &0.8491 &-0.0498 \\ \cline{2-4}
% &1.9 &0.8241 &-0.0491 \\ \cline{2-4}
% &1.9 &0.8245 &-0.0492 \\ \cline{2-4}
% &2.0 &0.7999 &-0.0480 \\ \hline
% \multicolumn{4}{|c|}{$y_5 = 0.7999961579105562$} \\ \hline

% \end{longtable}

% Vậy nghiệm gần đúng của phương trình là $y_5 = 0.7999961579105562$.\par

% c) $y'=x-\sin y$, $y(0)=0$, $h=0.1$.\par
% Ta lập được sơ đồ tính toán như sau:

% \begin{longtable}{|c|c|c|c|}\hline
% $n$&$x_o$ &$y_o$ &$hf(x_n,y_n)$ \\ \hline
% \endhead

% \multirow{4}{*}{$0$}
% &0.0 &0.0000 &0.0000 \\ \cline{2-4}
% &0.05 &0.0000 &0.0050 \\ \cline{2-4}
% &0.05 &0.0025 &0.0048 \\ \cline{2-4}
% &0.1 &0.0048 &0.0095 \\ \hline
% \multicolumn{4}{|c|}{$y_1 = 0.004837500380164618$} \\ \hline

% \multirow{4}{*}{$1$}
% &0.1 &0.0048 &0.0095 \\ \cline{2-4}
% &0.15 &0.0096 &0.0140 \\ \cline{2-4}
% &0.15 &0.0119 &0.0138 \\ \cline{2-4}
% &0.2 &0.0187 &0.0181 \\ \hline
% \multicolumn{4}{|c|}{$y_2 = 0.018730933533350227$} \\ \hline

% \multirow{4}{*}{$2$}
% &0.2 &0.0187 &0.0181 \\ \cline{2-4}
% &0.25 &0.0278 &0.0222 \\ \cline{2-4}
% &0.25 &0.0298 &0.0220 \\ \cline{2-4}
% &0.3 &0.0407 &0.0259 \\ \hline
% \multicolumn{4}{|c|}{$y_3 = 0.040818909363427004$} \\ \hline

% \multirow{4}{*}{$3$}
% &0.3 &0.0408 &0.0259 \\ \cline{2-4}
% &0.35 &0.0538 &0.0296 \\ \cline{2-4}
% &0.35 &0.0556 &0.0294 \\ \cline{2-4}
% &0.4 &0.0703 &0.0330 \\ \hline
% \multicolumn{4}{|c|}{$y_4 = 0.07032359540590269$} \\ \hline

% \multirow{4}{*}{$4$}
% &0.4 &0.0703 &0.0330 \\ \cline{2-4}
% &0.45 &0.0868 &0.0363 \\ \cline{2-4}
% &0.45 &0.0885 &0.0362 \\ \cline{2-4}
% &0.5 &0.1065 &0.0394 \\ \hline
% \multicolumn{4}{|c|}{$y_5 = 0.10654526828749952$} \\ \hline

% \multirow{4}{*}{$5$}
% &0.5 &0.1065 &0.0394 \\ \cline{2-4}
% &0.55 &0.1262 &0.0424 \\ \cline{2-4}
% &0.55 &0.1278 &0.0423 \\ \cline{2-4}
% &0.6 &0.1488 &0.0452 \\ \hline
% \multicolumn{4}{|c|}{$y_6 = 0.1488587139227904$} \\ \hline

% \multirow{4}{*}{$6$}
% &0.6 &0.1489 &0.0452 \\ \cline{2-4}
% &0.65 &0.1714 &0.0479 \\ \cline{2-4}
% &0.65 &0.1728 &0.0478 \\ \cline{2-4}
% &0.7 &0.1967 &0.0505 \\ \hline
% \multicolumn{4}{|c|}{$y_7 = 0.19671114595560976$} \\ \hline

% \multirow{4}{*}{$7$}
% &0.7 &0.1967 &0.0505 \\ \cline{2-4}
% &0.75 &0.2219 &0.0530 \\ \cline{2-4}
% &0.75 &0.2232 &0.0529 \\ \cline{2-4}
% &0.8 &0.2496 &0.0553 \\ \hline
% \multicolumn{4}{|c|}{$y_8 = 0.24962126239186969$} \\ \hline

% \multirow{4}{*}{$8$}
% &0.8 &0.2496 &0.0553 \\ \cline{2-4}
% &0.85 &0.2773 &0.0576	 \\ \cline{2-4}
% &0.85 &0.2784 &0.0575 \\ \cline{2-4}
% &0.9 &0.3071 &0.0598 \\ \hline
% \multicolumn{4}{|c|}{$y_9 = 0.3071790988049683$} \\ \hline

% \multirow{4}{*}{$9$}
% &0.9 &0.3072 &0.0598 \\ \cline{2-4}
% &0.95 &0.3371 &0.0619 \\ \cline{2-4}
% &0.95 &0.3381 &0.0618 \\ \cline{2-4}
% &1.0 &0.3690 &0.0639 \\ \hline
% \multicolumn{4}{|c|}{$y_{10} = 0.36904642670522175$} \\ \hline

% \multirow{4}{*}{$10$}
% &1.0 &0.3690 &0.0639 \\ \cline{2-4}
% &1.05 &0.4010 &0.0660 \\ \cline{2-4}
% &1.05 &0.4020 &0.0659 \\ \cline{2-4}
% &1.1 &0.4349 &0.0679 \\ \hline
% \multicolumn{4}{|c|}{$y_{11} = 0.43495756238415795$} \\ \hline

% \multirow{4}{*}{$11$}
% &1.1 &0.4350 &0.0679 \\ \cline{2-4}
% &1.15 &0.4689 &0.0698 \\ \cline{2-4}
% &1.15 &0.4699 &0.0697 \\ \cline{2-4}
% &1.2 &0.5047 &0.0716 \\ \hline
% \multicolumn{4}{|c|}{$y_{12} = 0.5047205776597383$} \\ \hline

% \multirow{4}{*}{$12$}
% &1.2 &0.5047 &0.0716 \\ \cline{2-4}
% &1.25 &0.5405 &0.0735 \\ \cline{2-4}
% &1.25 &0.5415 &0.0735 \\ \cline{2-4}
% &1.3 &0.5782 &0.0754 \\ \hline
% \multicolumn{4}{|c|}{$y_{13} = 0.5782190292422908$} \\ \hline

% \multirow{4}{*}{$13$}
% &1.3 &0.5782 &0.0753 \\ \cline{2-4}
% &1.35 &0.6159 &0.0772 \\ \cline{2-4}
% &1.35 &0.6168 &0.0772 \\ \cline{2-4}
% &1.4 &0.6554 &0.0791 \\ \hline
% \multicolumn{4}{|c|}{$y_{14} = 0.6554144439285761$} \\ \hline

% \multirow{4}{*}{$14$}
% &1.4 &0.6554 &0.0791 \\ \cline{2-4}
% &1.45 &0.6949 &0.0810 \\ \cline{2-4}
% &1.45 &0.6959 &0.0809 \\ \cline{2-4}
% &1.5 &0.7363 &0.0828 \\ \hline
% \multicolumn{4}{|c|}{$y_{15} = 0.736349914185467$} \\ \hline

% \end{longtable}

% Vậy nghiệm gần đúng của phương trình đã cho là $y_{15} = 0.736349914185467$.\par

% d) $y'=x^2+y^2$, $y(0)=1$, $h=0.1$.\par
% Ta lập được sơ đồ tính toán như sau:

% \begin{longtable}{|c|c|c|c|}\hline
% $n$&$x_o$ &$y_o$ &$hf(x_n,y_n)$ \\ \hline
% \endhead

% \multirow{4}{*}{$0$}
% &0. 0 000 &1.0000 &0.1000\\ \cline{2-4}
% &0.0500 &1.0500 &0.1105\\ \cline{2-4}
% &0.0500 &1.0553 &0.1116\\ \cline{2-4}
% &0.1000 &1.1116 &0.1246\\ \hline
% \multicolumn{4}{|c|}{$y_1 = 1.1114628561787105$}\\ \hline

% \multirow{4}{*}{$1$}
% &0.1000 &1.1115 &0.1245\\ \cline{2-4}
% &0.1500 &1.1737 &0.1400\\ \cline{2-4}
% &0.1500 &1.1815 &0.1418\\ \cline{2-4}
% &0.2000 &1.2533 &0.1611\\ \hline
% \multicolumn{4}{|c|}{$y_2 = 1.2530151746035345$}\\ \hline

% \multirow{4}{*}{$2$}
% &0.2000 &1.2530 &0.1610\\ \cline{2-4}
% &0.2500 &1.3335 &0.1841\\ \cline{2-4}
% &0.2500 &1.3451 &0.1872\\ \cline{2-4}
% &0.3000 &1.4402 &0.2164\\ \hline
% \multicolumn{4}{|c|}{$y_3 = 1.439665974547582$}\\ \hline

% \multirow{4}{*}{$3$}
% &0.3000 &1.4397 &0.2163\\ \cline{2-4}
% &0.3500 &1.5478 &0.2518\\ \cline{2-4}
% &0.3500 &1.5656 &0.2574\\ \cline{2-4}
% &0.4000 &1.6970 &0.3040\\ \hline
% \multicolumn{4}{|c|}{$y_4 = 1.696097903722817$}\\ \hline

% \multirow{4}{*}{$4$}
% &0.4000 &1.6961 &0.3037\\ \cline{2-4}
% &0.4500 &1.8479 &0.3617\\ \cline{2-4}
% &0.4500 &1.8770 &0.3726	\\ \cline{2-4}
% &0.5000 &2.0686 &0.4529\\ \hline
% \multicolumn{4}{|c|}{$y_5 = 2.066961015450312$}\\ \hline

% \multirow{4}{*}{$5$}
% &0.5000 &2.0670 &0.4522\\ \cline{2-4}
% &0.5500 &2.2931 &0.5561\\ \cline{2-4}
% &0.5500 &2.3450 &0.5802\\ \cline{2-4}
% &0.6000 &2.6471 &0.7367\\ \hline
% \multicolumn{4}{|c|}{$y_6 = 2.643860197079332$}\\ \hline

% \multirow{4}{*}{$6$}
% &0.6000 &2.6439 &0.7350\\ \cline{2-4}
% &0.6500 &3.0114 &0.9491\\ \cline{2-4}
% &0.6500 &3.1184 &1.0147\\ \cline{2-4}
% &0.7000 &3.6586 &1.3875\\ \hline
% \multicolumn{4}{|c|}{$y_7 = 3.6522003587842087$}\\ \hline

% \multirow{4}{*}{$7$}
% &0.7000 &3.6522 &1.3829\\ \cline{2-4}
% &0.7500 &4.3436 &1.9430\\ \cline{2-4}
% &0.7500 &4.6237 &2.1941\\ \cline{2-4}
% &0.8000 &5.8463 &3.4819\\ \hline
% \multicolumn{4}{|c|}{$y_8 = 5.842013335279946$}\\ \hline

% \multirow{4}{*}{$8$}
% &0.8000 &5.8420 &3.4769\\ \cline{2-4}
% &0.8500 &7.5805 &5.8186\\ \cline{2-4}
% &0.8500 &8.7513 &7.7308\\ \cline{2-4}
% &0.9000 &13.5728 &18.5031\\ \hline
% \multicolumn{4}{|c|}{$y_9 = 14.021820076380461$}\\ \hline

% \multirow{4}{*}{$9$}
% &0.9000 &14.0218 &19.7421\\ \cline{2-4}
% &0.9500 &23.8929 &57.1773\\ \cline{2-4}
% &0.9500 &42.6105 &181.6554\\ \cline{2-4}
% &1.0000 &195.6772 &3829.0565\\ \hline
% \multicolumn{4}{|c|}{$y_{10} = 735.0991433436242$}\\ \hline

% \end{longtable}

% Vậy nghiệm gần đúng của phương trình đã cho là $y_{10} = 735.0991433436242$.\par

% e) $y'=x-\sqrt{y}$, $y(0)=1$, $h=0.2$.\par
% Ta lập được sơ đồ tính toán như sau:

% \begin{longtable}{|c|c|c|c|}\hline
% $n$ & $x_o$ &$y_o$ &$hf(x_n,y_n)$ \\ \hline
% \endhead

% \multirow{4}{*}{$0$}
% &0.0000 &1.0000 &-0.2000\\ \cline{2-4}
% &0.1000 &0.9000 &-0.1697\\ \cline{2-4}
% &0.1000 &0.9151 &-0.1713\\ \cline{2-4}
% &0.2000 &0.8287 &-0.1421\\ \hline
% \multicolumn{4}{|c|}{$y_1 = 0.8293022416154283$}\\ \hline

% \multirow{4}{*}{$1$}
% &0.2000 &0.8293 &-0.1421\\ \cline{2-4}
% &0.3000 &0.7582 &-0.1142\\ \cline{2-4}
% &0.3000 &0.7722 &-0.1158\\ \cline{2-4}
% &0.4000 &0.7135 &-0.0889\\ \hline
% \multicolumn{4}{|c|}{$y_2 = 0.71415419659681$}\\ \hline

% \multirow{4}{*}{$14$}
% &0.4000 &0.7142 &-0.0890\\ \cline{2-4}
% &0.5000 &0.6696 &-0.0637\\ \cline{2-4}
% &0.5000 &0.6823 &-0.0652\\ \cline{2-4}
% &0.6000 &0.6489 &-0.0411\\ \hline
% \multicolumn{4}{|c|}{$y_3 = 0.6495093765790388$}\\ \hline

% \multirow{4}{*}{$3$}
% &0.6000 &0.6495 &-0.0412\\ \cline{2-4}
% &0.7000 &0.6289 &-0.0186\\ \cline{2-4}
% &0.7000 &0.6402 &-0.0200\\ \cline{2-4}
% &0.8000 &0.6295 &0.0013\\ \hline
% \multicolumn{4}{|c|}{$y_4 = 0.6299872457446561$}\\ \hline

% \multirow{4}{*}{$4$}
% &0.8000 &0.6300 &0.0013\\ \cline{2-4}
% &0.9000 &0.6306 &0.0212\\ \cline{2-4}
% &0.9000 &0.6406 &0.0199\\ \cline{2-4}
% &1.0000 &0.6499 &0.0388\\ \hline
% \multicolumn{4}{|c|}{$y_5 = 0.6503593651753093$}\\ \hline

% \end{longtable}

% Vậy ta có nghiệm gần đúng của phương trình đã cho là $y_5 = 0.6503593651753093$.\par

% \textbf{Câu 7.} Tìm nghiệm gần đúng của các phương trình sau bằng phương pháp nội suy Adams và ngoại suy Adams tương ứng với $k=4$ và $k=3$ biết các giá trị đầu tiên được tìm bằng phương pháp RK4.
% \begin{enumerate}[label=\alph*)]
% 	\item $y'=xy^3-y$, $y(0)=1$, $h=0.1$ trên đoạn $[0;1]$
% 	\item $y'=x-y$, $y(0)=1$, $h=0.1$ trên đoạn $[0;1]$
% 	\item $y'=1-x\sqrt[3]{y}$, $y(0)=1$, $h=0.5$ trên đoạn $[0;5]$
% 	\item $y'=y-x^2+1$, $y(0)=0.5$, $h=0.2$ trên đoạn $[0;2]$
% 	\item $y'=xe^{3x}-2y$, $y(0)=0$, $h=0.2$ trên đoạn $[0;1]$
% \end{enumerate}

% \textbf{Giải}\par

% a) $y'=xy^3-y$, $y(0)=1$, $h=0.1$\par

% $\star$~\textit{Phương pháp nội suy Adams}\par
% Ta lập được bảng tính toán nghiệm gần đúng theo phương pháp nội suy Adams với $k=4$ như sau:

% \begin{longtable}{|c|c|c|c|}\hline
% $i$&$x_i$&Adams&RK4\\ \hline
% \endhead
% 4  & 0.4 & 0.704867563 & 0.704859905 \\ \hline
% 5  & 0.5 & 0.651066638 & 0.651062961 \\ \hline
% 6  & 0.6 & 0.601924692 & 0.601922792 \\ \hline
% 7  & 0.7 & 0.556622667 & 0.556621625 \\ \hline
% 8  & 0.8 & 0.514582197 & 0.514581597 \\ \hline
% 9  & 0.9 & 0.475392441 & 0.475392079 \\ \hline
% 10 & 1.0 & 0.438760084 & 0.438759854 \\ \hline
% \end{longtable}


% $\star$~\textit{Phương pháp ngoại suy Adams}\par
% % Ta có $y'=xy^3-y$, $y(0)=1$, $h=0.1$\par
% Ta lập được bảng tính toán nghiệm gần đúng theo phương pháp ngoại suy Adams với $k=3$ như sau:

% \begin{longtable}{|c|c|c|c|}
% \hline
% $i$&$x_i$&Adams&RK4\\ \hline
% \endhead
% 4 &0.4 &0.705130139 &0.704859905\\ \hline
% 5 &0.5 &0.651205954 &0.651062961\\ \hline
% 6 &0.6 &0.602004109 &0.601922792\\ \hline
% 7 &0.7 &0.556670805 &0.556621625\\ \hline
% 8 &0.8 &0.514613013 &0.514581597\\ \hline
% 9 &0.9 &0.475413151 &0.475392079\\ \hline
% 10 &1.0 &0.438774587 &0.438759854\\ \hline
% \end{longtable}

% b) $y'=x-y$, $y(0)=1$, $h=0.1$\par

% $\star$~\textit{Phương pháp nội suy Adams}\par
% Ta lập được bảng tính toán nghiệm gần đúng theo phương pháp nội suy Adams với $k=4$ như sau:

% \begin{longtable}{|c|c|c|c|}\hline
% $i$&$x_i$&Adams&RK4\\ \hline
% \endhead
% 4  &0.4 &0.740640480 &0.740640578\\ \hline
% 5  &0.5 &0.713061780 &0.713061869\\ \hline
% 6  &0.6 &0.697623789 &0.697623869\\ \hline
% 7  &0.7 &0.693171165 &0.693171237\\ \hline
% 8  &0.8 &0.698658514 &0.698658579\\ \hline
% 9  &0.9 &0.713139923 &0.713139982\\ \hline
% 10 &1.0 &0.735759495 &0.735759549\\ \hline
% \end{longtable}

% $\star$~\textit{Phương pháp ngoại suy Adams}\par
% Ta có: $y'=x-y$, $y(0)=1$, $h=0.1$\par
% Ta lập được bảng tính toán nghiệm gần đúng theo phương pháp ngoại suy Adams với $k=3$ như sau:

% \begin{longtable}{|c|c|c|c|}\hline
% $i$&$x_i$&Adams&RK4\\ \hline
% \endhead
% 4  &0.4 &0.740646198 &0.740640578\\ \hline
% 5  &0.5 &0.713066954 &0.713061869\\ \hline
% 6  &0.6 &0.697628470 &0.697623869\\ \hline
% 7  &0.7 &0.693175401 &0.693171237\\ \hline
% 8  &0.8 &0.698662347 &0.698658579\\ \hline
% 9  &0.9 &0.713143391 &0.713139982\\ \hline
% 10 &1.0 &0.735762633 &0.735759549\\ \hline
% \end{longtable}


% d) $y'=y-x^2+1$, $y(0)=0.5$, $h=0.2$\par

% $\star$~\textit{Phương pháp nội suy Adams}\par
% Ta lập được bảng tính toán nghiệm gần đúng theo phương pháp nội suy Adams với $k=3$ như sau:

% \begin{longtable}{|c|c|c|c|}\hline
% $i$&$x_i$&Adams&RK4\\ \hline
% \endhead
% 4  &0.8 &2.127205481 &2.127202685\\ \hline
% 5  &1.0 &2.640824775 &2.640822693\\ \hline
% 6  &1.2 &3.179895380 &3.179894170\\ \hline
% 7  &1.4 &3.732340217 &3.732340073\\ \hline
% 8  &1.6 &4.283408341 &4.283409498\\ \hline
% 9  &1.8 &4.815082947 &4.815085695\\ \hline
% 10 &2.0 &5.305358312 &5.305363001\\ \hline
% \end{longtable}

% $\star$~\textit{Phương pháp ngoại suy Adams}\par
% % Ta có $y'=y-x^2+1$, $y(0)=0.5$, $h=0.2$\par
% Ta lập được bảng tính toán nghiệm gần đúng theo phương pháp ngoại suy Adams với $k=4$ như sau:

% \begin{longtable}{|c|c|c|c|}
% \hline
% $i$&$x_i$&Adams&RK4\\ \hline
% \endhead
% 4  &0.8 &2.127289249 &2.127202685\\ \hline
% 5  &1.0 &2.640927089 &2.640822693\\ \hline
% 6  &1.2 &3.180020346 &3.179894170\\ \hline
% 7  &1.4 &3.732492851 &3.732340073\\ \hline
% 8  &1.6 &4.283594768 &4.283409498\\ \hline
% 9  &1.8 &4.815310650 &4.815085695\\ \hline
% 10 &2.0 &5.305636428 &5.305363001\\ \hline
% \end{longtable}

% e) $y'=xe^{3x}-2y$, $y(0)=0$, $h=0.2$\par

% $\star$~\textit{Phương pháp nội suy Adams}\par
% Ta lập được bảng tính toán nghiệm gần đúng theo phương pháp nội suy Adams với $k=4$ như sau:

% \begin{longtable}{|c|c|c|c|}\hline
% $i$&$x_i$&Adams&RK4\\ \hline
% \endhead
% 4 &0.8 &1.332737617 &1.332227617\\ \hline
% 5 &1.0 &3.222889913 &3.221992603\\ \hline
% \end{longtable}

% $\star$~\textit{Phương pháp ngoại suy Adams}\par
% % Ta có: $y'=xe^{3x}-2y$, $y(0)=0$, $h=0.2$\par 
% Ta lập được bảng tính toán nghiệm gần đúng theo phương pháp ngoại suy Adams với $k=3$ như sau:

% \begin{longtable}{|c|c|c|c|}\hline
% $i$&$x_i$&Adams&RK4\\ \hline
% \endhead
% 4 &0.8 &1.296385456 &1.332227617 \\ \hline
% 5 &1.0 &3.149614338 &3.221992603 \\ \hline\end{longtable}
