\section{Phương pháp Euler và Euler cải tiến}
\subsection{Phương pháp Euler}

Xét bài toán Cauchy
\begin{equation}\label{eqn:eqn615}
	y'=f(x,y),~y(x_0)=y_0
\end{equation}

Giả sử hàm $f$ thỏa mãn điều kiện $|f(x,y_1)-f(x,y_2)|\leqslant L|{y_1}-{y_2}|$
và $\left|\frac{\mathrm{d}f}{\mathrm{d}x}\right|=\left|\frac{\partial f}{\partial x}+f\frac{\partial f}{\partial y}\right|\leqslant M$ trong hình chữ nhật:
$$D=\left\{(x,y)\in\mathbb{R}^2\Big||x-x_0|\leqslant a,|y-y_0|\leqslant b\right\}$$

Giả sử $x_0$ là giá trị ban đầu và $h$ là số dương cho trước đủ nhỏ với $x_i=x_0+ih$, với $i=0,1,2,\ldots,h$ được gọi là độ dài bước.\par
(\ref{eqn:eqn615}) tương đương với $\mathrm{d}y=f\left(x,y\right)\mathrm{d}x$, lấy tích phân hai vế ta được:
\begin{center}
	$\int\limits_{y_0}^{y_1}\mathrm{d}y=\int\limits_{x_0}^{x_1}\mathrm{d}x$ hay $y_1=y_0+\int\limits_{x_0}^{x_1} f(x,y)\mathrm{d}x$
\end{center}

Giả sử $f(x,y)\approx f\left(x_0,y_0\right)$ với $x_0\leqslant x\leqslant x_1$, khi đó:
\begin{center}
$y_1\approx y_0+f\left(x_0,y_0\right)\left(x_1-x_0\right)$ hay $y_1\approx y_0+hf\left(x_0,y_0\right)$
\end{center}

Tương tự, với $x_1\leqslant x\leqslant x_2$, ta có $y_2\approx y_1+hf\left(x_1,y_1\right)$.
Từ đó, ta có công thức tổng quát:\\
\begin{equation}\label{eqn:eqn616}
	y_{n+1}=y_n+hf\left(x_n,y_n\right),~n=0, 1, 2, \ldots
\end{equation}

Công thức (\ref{eqn:eqn616}) được gọi là phương pháp Euler.\\

\begin{example}
	Bằng phương pháp Euler, giải phương trình gần đúng phương trình:
	$$y'=2xy+\mathrm{e}^{x^2},~y(0)=1$$
	trong đoạn $[0;1,5]$, so sánh với nghiệm chính xác $\varphi(x)=(x+1)\mathrm{e}^{x^2}$ của phương trình.\par
\end{example}

\textbf{Giải}\par
Với $x_0=0,~y_0=1$, chọn $h=0.25$.\par
Áp dụng công thức (\ref{eqn:eqn616}), ta có bảng giá trị:\par
\begin{longtable}{|c|c|c|c|}\hline
	$n$ & $x_n$  & $y_n$     & $\varphi(x_n)$ \\\hline
	\endhead
	$0$ & $0.0$  & $1.0$     & $1.0$     \\\hline
	$1$ & $0.25$ & $1.25$    & $1.3306$  \\\hline
	$2$ & $0.5$  & $1.6724$  & $1.926$   \\\hline
	$3$ & $0.75$ & $2.4115$  & $3.0713$  \\\hline
	$4$ & $1.0$  & $3.7545$  & $5.4366$  \\\hline
	$5$ & $1.25$ & $6.3114$  & $10.7341$ \\\hline
	$6$ & $1.5$  & $11.4487$ & $23.7193$ \\\hline
\end{longtable}

Nếu ta chọn $h=0.1$, ta cũng có bảng giá trị:\par
\begin{longtable}{|c|c|c|c||c|c|c|c|}\hline
	$n$ & $x_n$ & $y_n$    & $\varphi(x_n)$ & $n$  & $x_n$  & $y_n$      & $\varphi(x_n)$\\ \hline
	\endhead
	$0$ & $0.0$ & $1.0$    & $1.0$          & $8$  & $0.8$  & $3.0338$  & $3.4137$      \\ \hline
	$1$ & $0.1$ & $1.1$    & $1.1111$       & $9$  & $0.9$  & $3.7088$  & $4.271$       \\ \hline
	$2$ & $0.2$ & $1.223$  & $1.249$        & $10$ & $1.0$  & $4.6012$  & $5.4366$      \\ \hline
	$3$ & $0.3$ & $1.376$  & $1.4224$       & $11$ & $1.1$  & $5.7933$  & $7.0423$      \\ \hline
	$4$ & $0.4$ & $1.568$  & $1.6429$       & $12$ & $1.2$  & $7.4031$  & $9.2855$      \\ \hline
	$5$ & $0.5$ & $1.8108$ & $1.926$        & $13$ & $1.3$  & $9.602$   & $12.4648$     \\ \hline
	$6$ & $0.6$ & $2.1203$ & $2.2933$       & $14$ & $1.4$  & $12.6404$ & $17.0384$     \\ \hline
	$7$ & $0.7$ & $2.518$  & $2.7749$       & $15$ & $1.5$  & $16.8897$ & $23.7193$     \\ \hline
\end{longtable}

\begin{example}
	Bằng phương pháp Euler, giải gần đúng phương trình vi phân:
	$$y'+2y=2-{\mathrm{e}^{-4x}},y(0)=1$$
	với $h=0.1$ cho trước.\par
\end{example}

\textbf{Giải}\par
Phương trình vi phân đã cho có nghiệm chính xác là:\par
$$y=1+\frac{1}{2}{\mathrm{e}^{-4x}}-\frac{1}{2}{\mathrm{e}^{-2x}}$$

Phương trình vi phân đã cho tương đương với: $y'=2-\mathrm{e}^{-4x}-\frac12\mathrm{e}^{-2x}-2y$.\par
Với $x_0=0,~y_0=1$ và $h=0.1$. Khi đó ta có bảng giá trị:
\begin{longtable}{|c|c|c|c|}\hline
	$n$ & $x_n$ & $y_n$      & $\varphi(x_n)$ \\\hline
	\endhead
	$0$ & $0.0$ & $1.0$      & $1.0$      \\\hline
	$1$ & $0.1$ & $0.9$      & $0.925795$ \\\hline
	$2$ & $0.2$ & $0.852968$ & $0.889504$ \\\hline
	$3$ & $0.3$ & $0.837441$ & $0.876191$ \\\hline
	$4$ & $0.4$ & $0.839834$ & $0.876284$ \\\hline
	$5$ & $0.5$ & $0.851677$ & $0.883728$ \\\hline
\end{longtable}

Nếu ta thay đổi độ dài bước lần lượt là $h=0.05$, $h=0.01$, $h=0.005$ và $h=0.001$, ta có bảng giá trị của nghiệm gần đúng và nghiệm chính xác tại các thời điểm $x=1$, $x=2$, $x=3$, $x=4$ và $x=5$ là:\par
\begin{longtable}{|c|c|c|c|c|c|}\hline
	\multirow{2}{*}{$x$} & \multirow{2}{*}{Nghiệm chính xác} & \multicolumn{4}{c|}{Nghiệm gần đúng}\\\cline{3-6}
	    &             & $h=0.05$    & $h=0.01$    & $h=0.005$   & $h=0.001$   \\\hline
	\endhead
	$1$ & $0.9414902$ & $0.9364698$ & $0.9404994$ & $0.9409957$ & $0.9413914$ \\\hline
	$2$ & $0.9910099$ & $0.9911126$ & $0.9910193$ & $0.9910139$ & $0.9910106$ \\\hline
	$3$ & $0.9987637$ & $0.9988982$ & $0.9987890$ & $0.9987763$ & $0.9987662$ \\\hline
	$4$ & $0.9998323$ & $0.9998657$ & $0.9998390$ & $0.9998357$ & $0.9998330$ \\\hline
	$5$ & $0.9999773$ & $0.9999837$ & $0.9999786$ & $0.9999780$ & $0.9999774$ \\\hline
\end{longtable}

Phần trăm sai số giữa nghiệm chính xác và nghiệm gần đúng được xác định bởi:\par
\begin{equation}\label{eqn:eqn617}
	P=\frac{|n_\text{exact}-n_\text{approx}|}{n_\text{exact}}\times 100\%
\end{equation}

Áp dụng (\ref{eqn:eqn617}), ta có bảng tính toán phần trăm sai số giữa nghiệm chính xãc và nghiệm gần đúng:
\begin{longtable}{|c|c|c|c|c|}\hline
	$x$ & $h=0.05$    & $h=0.01$    & $h=0.005$    & $h=0.001$    \\\hline
	\endhead
	$1$ & $0.53\%$    & $0.105\%$   & $0.053\%$    & $0.0105\%$   \\\hline
	$2$ & $0.01\%$    & $0.0094\%$  & $0.00041\%$  & $0.00007\%$  \\\hline
	$3$ & $0.013\%$   & $0.0025\%$  & $0.0013\%$   & $0.00025\%$  \\\hline
	$4$ & $0.0033\%$  & $0.00067\%$ & $0.00034\%$  & $0.000067\%$ \\\hline
	$5$ & $0.00064\%$ & $0.00013\%$ & $0.000068\%$ & $0.000014\%$ \\\hline
\end{longtable}

\begin{remark}~\begin{itemize}
	\item Nếu độ dài bước $h$ càng nhỏ, sai số giữa nghiệm gần đúng và nghiệm chính xác càng nhỏ.
	\item Nói chung sai số giữa nghiệm gần đúng và nghiệm chính xác sẽ tăng nếu giá trị $x$ tăng.
\end{itemize}\end{remark}

% \subsection{Phương pháp Euler cải tiến}
\subsection{Phương pháp Euler cải tiến thứ nhất}

Phương pháp Euler cải tiến thứ nhất là một phương pháp hiện trong đó giá trị tiếp theo của nghiệm, $y_{n+1}$ được thực hiện thông qua việc tính toán các giá trị trung gian:\par
\begin{equation}\label{eqn:eqn618}\begin{cases}
	x_{n+\frac12}&=x_n+\frac h2\\
	f_{n+\frac12}&=f\left(x_n+\frac h2, y_n+\frac h2f_n\right)\\
	y_{n+1}&=y_n+hf_{n+\frac12}
\end{cases}\end{equation}

\begin{example}
	Áp dụng phương pháp Euler cải tiến thứ nhất, giải phương trình vi phân sau:
	\begin{equation}
		y'=\frac{2y}{x}+x,~y(1)=1
	\end{equation}
	trên đoạn $\left[1;1.4\right]$ và độ dài bước $h=0.1$.\par
\end{example}

\textbf{Giải}\par
Phương trình đã cho có nghiệm gần đúng là $\varphi(x)=x^2+x^2\ln x$.\par
Đặt $f(x,y)=\frac{2y}{x}+x$, ta có bảng giá trị:
\begin{longtable}{|c|c|c|c|}\hline
	$n$ & $x_n$ & $y_n$      & $\varphi(x_n)$\\\hline
	\endhead
	$0$ & $1.0$ & $1.0$      & $1.0$      \\\hline
	$1$ & $1.1$ & $1.324048$ & $1.325325$ \\\hline
	$2$ & $1.2$ & $1.699816$ & $1.702543$ \\\hline
	$3$ & $1.3$ & $2.12905$  & $2.133396$ \\\hline
	$4$ & $1.4$ & $2.613357$ & $2.619486$ \\\hline
\end{longtable}

\subsection{Phương pháp Euler cải tiến thứ hai}

Từ (\ref{eqn:eqn616}) ta có:
$$y_{n+1}=y_n+hf(x_n,y_n)=y_n+\frac h2\big[f(x_n,y_n)+f(x_n,y_n )\big]$$

Thay số hạng thứ hai $f(x_n,y_n)$ bởi $f(x_{n+1},y_{n+1})$, ta được:
$$y_{n+1}=y_n+hf(x_n,y_n)=y_n+\frac h2\left[ f(x_n,y_n)+f(x_{n+1},y_{n+1}) \right]$$

Tuy nhiên giá trị $y_{n+1}$ bên vế phải cũng là giá trị cần tính, vì vậy ta tính $f(x_{n+1},y_{n+1})$ với $y_{n+1}$ được tính bởi phương pháp Euler (\ref{eqn:eqn616}). Khi đó ta có một phương pháp mới gọi là phương pháp Euler cải tiến thứ hai.\par
Để thực hiện việc tính giá trị $y_{n+1}$  khi biết $y_n$  ta thực hiện các bước tính toán:\par

\begin{equation}\label{eqn:eqn619}\begin{cases}
	\tilde{y}_{n+1}&=y_n+hf(x_n,y_n)\\
	 y_{n+1}&=y_n+\frac h2\left[ f(x_n,y_n)+f(x_{n+1},y_{n+1}) \right].
\end{cases}\end{equation}

\begin{example}
	Áp dụng phương pháp Euler và Euler cải tiến thứ hai, giải phương trình vi phân sau:
	$$y'=y-x,~y(0)=\frac12$$
	trên đoạn $[0,1]$ với độ dài bước $h=0.1$.\par
\end{example}

\textbf{Giải}\par
Phương trình đã cho có nghiệm gần đúng của phương trình vi phân đã cho có là $\varphi(x)=x+1-\frac12\mathrm{e}^x$.\par

Với $h=0.1$ và $x_n=x_0+nh=nh$.\par

Theo (\ref{eqn:eqn616}) nghiệm gần đúng của phương trình đã cho được tính theo phương pháp Euler:
$$\begin{cases}
	y_0&=y(0) = 1\\
	y_{n+1}& = y_n+hf(x_n,y_n)=y_n+0.1(y_n-x_n)=1.1y_n-0.1x_n
\end{cases}$$

Theo (\ref{eqn:eqn619}) nghiệm gần đúng của phương trình đã cho được tính theo phương pháp Euler cải tiến:
$$\begin{cases}
	y_0&=y(0)=1\\
	\tilde{y}_{n+1}&=y_n + hf(x_n,y_n) = y_n + 0.1 (y_n-x_n) = 1,1y_n - 0.1x_n,\\
	y_{n+1}&=y_n+\frac h2[f(x_n,y_n)+f(x_{n+1}, \tilde{y}_{n+1})]=y_n+\frac{0.1}{2}[(y_n-x_n)+(\tilde{y}_{n+1}-x_n)]\\
	~&=y_n+\frac{0.1}{2}[(y_n-x_n)+(1,1y_n-0.1x_n-x_n)]=1,06y_n-0.105x_n
\end{cases}$$

Từ đó ta có bảng nghiệm giá trị theo hai phương pháp:\par
\begin{longtable}{|c|c|c|c|c|c|c|}\hline
	$n$  & $x_n$ & $\tilde{y}_n$ & $y_{n}$ (Euler cải tiến) & $\varphi(x_n)$ \\ \hline
	\endhead
	$0$  & $0.0$ & $0.5$         & $0.5$      & $0.5$      \\ \hline
	$1$  & $0.1$ & $0.55$        & $0.5475$   & $0.547415$ \\ \hline
	$2$  & $0.2$ & $0.595$       & $0.589625$ & $0.589299$ \\ \hline
	$3$  & $0.3$ & $0.6345$      & $0.625831$ & $0.625071$ \\ \hline
	$4$  & $0.4$ & $0.66795$     & $0.65552$  & $0.654088$ \\ \hline
	$5$  & $0.5$ & $0.694745$    & $0.678034$ & $0.675639$ \\ \hline
	$6$  & $0.6$ & $0.71422$     & $0.692646$ & $0.688941$ \\ \hline
	$7$  & $0.7$ & $0.725641$    & $0.698561$ & $0.693124$ \\ \hline
	$8$  & $0.8$ & $0.728206$    & $0.694899$ & $0.68723$  \\ \hline
	$9$  & $0.9$ & $0.721026$    & $0.680695$ & $0.670198$ \\ \hline
	$10$ & $1.0$ & $0.703129$    & $0.654886$ & $0.640859$ \\ \hline
\end{longtable}
Qua bảng giá trị, ta thấy phương pháp Euler cải tiến cho nghiệm tốt hơn phương pháp Euler.

\begin{example}
	Sử dụng phương pháp Euler và Euler cải tiến, giải gần đúng phương trình $$y'=\frac{y^2-x}{1+x}$$
	trên đoạn $[0;0.2]$, với $y(0)=1$ và $h=0.02$.
\end{example}

\textbf{Giải}\par
Với $x_0=0$, $y_0=1$ và $h=0.02$, ta có bảng giá trị:

\begin{center}
\begin{tabular}{|c|c|c|c|c|}\hline
	$n$  & $x_n$  & $y_n$      & $y_n$ (Euler cải tiến 1) & $y_n$ (Euler cải tiến 2) \\\hline
	% \endhead
	$0$  & $0.0$  & $1.0$      & $1.0$      & $1.0$      \\\hline
	$1$  & $0.02$ & $1.02$     & $1.020002$ & $1.020004$ \\\hline
	$2$  & $0.04$ & $1.040008$ & $1.04002$  & $1.040023$ \\\hline
	$3$  & $0.06$ & $1.060039$ & $1.060069$ & $1.060074$ \\\hline
	$4$  & $0.08$ & $1.080108$ & $1.080164$ & $1.080171$ \\\hline
	$5$  & $0.1$  & $1.100231$ & $1.100322$ & $1.100329$ \\\hline
	$6$  & $0.12$ & $1.120422$ & $1.120557$ & $1.120564$ \\\hline
	$7$  & $0.14$ & $1.140696$ & $1.140884$ & $1.140889$ \\\hline
	$8$  & $0.16$ & $1.161068$ & $1.161318$ & $1.161321$ \\\hline
	$9$  & $0.18$ & $1.181552$ & $1.181875$ & $1.181874$ \\\hline
	$10$ & $0.2$  & $1.202163$ & $1.202569$ & $1.202563$ \\\hline
\end{tabular}
\end{center}
