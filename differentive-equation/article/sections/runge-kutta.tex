\section{Phương pháp Runge – Kutta}
\subsection{Nội dung phương pháp}

Xét bài toán Cauchy (\ref{eqn:eqn603}), (\ref{eqn:eqn604}), để giải bài toán này, xuất pháp từ giá trị $y_n$ ta tìm được giá trị gần đúng $y_{n+1}$ tại $x_{n+1}=x_n + h$ theo công thức:\\
\begin{equation}\label{eqn:eqn620}
	y_{n+1}=y_n+\sum\limits_{i=1}^s b_i k_i
\end{equation}
trong đó\\
\begin{equation}\label{eqn:eqn621}
	k_i=hf\left(x_n+c_ih, y_n+h\sum\limits_{j=1}^s a_{ij} k_j\right)
\end{equation}

Công thức (\ref{eqn:eqn620}) và (\ref{eqn:eqn621}) xác định phương pháp Runge – Kutta tổng quát.\par

Các hệ số $c_i$, $a_{ij}$, $b_i$ được chọn sao cho với $m$ đủ lớn, hàm số $\varphi(h)=y(x_n+h)-y_n-\sum\limits_{i=1}^s b_i k_i$ thỏa mãn
\begin{equation}\label{eqn:eqn622}
	\varphi(0)=\varphi'(0)=\varphi''(0)=\ldots=\varphi^{(m)}(0)=0;~\varphi^{(m+1)}\neq 0
\end{equation}

Khi đó sai số trong mỗi bước được đánh giá bởi:\\
\begin{equation}\label{eqn:eqn623}
	R(h)=\frac{\varphi^{(m+1)} \left(\xi\right)}{(m+1)!}h^{(m+1)},~0<\xi<h
\end{equation}

Từ (\ref{eqn:eqn622}), với $l=0,~1,~2,\ldots,~m$ ta rút ra:
\begin{equation}\label{eqn:eqn624}
	y^{(l)}_n=\sum\limits_{i=1}^s b_i k^{(l)}_i (0)
\end{equation}

Ta xét một số trường hợp đặc biệt:
\begin{enumerate}[label=\alph*)]
	\item Với $s=1$:\par
		Theo (\ref{eqn:eqn621}), $\varphi(h)=y(x_n+h)-y(x_n)-b_1hf(x_n,y_n)$.\par
		Nên $\varphi' (h)= y'(x_n+h)-b_1f(x_n,y_n)=(1-b_1)f(x_n,y_n)$.\par
		Ta thấy $\varphi'(h)=0 $ với mọi $f$ khi và chỉ khi $b_1=1$. Từ đó, công thức Runge – Kutta khi $s=1$ là:\par
		\begin{equation}\label{enq:eqn625}
			y_{n+1} = y_n + hf(x_n,y_n)
		\end{equation}
		Rõ ràng (\ref{enq:eqn625}) là công thức Euler.
	\item Với $s=2$:\par
		Theo (\ref{eqn:eqn622}) và (\ref{eqn:eqn624}), ta cũng có:
		\begin{align*}
			k_1 &= hf(x_n,y_n)\\
			k_2 &= hf(x_n+c_2 h,y_n+a_{21} k_1)\\
			\varphi(h) &= y(x_n+h)-y_n-b_1 k_1-b_2 k_2\\
			y_n^{(l)} &= b_1 k_1^{(l)}(0)+b_2 k_2^{(l)}(0)
		\end{align*}
		Ta có: $k'_1(h)= f(x_n,y_n)\Rightarrow k'_1 (0)=f(x_n,y_n)$.\par
		$k'_2(h)=f(x_n+c_2 h,y_n+a_{21} k_1)+h\left[\frac{\partial f}{\partial x}c_2+\frac{\partial f}{\partial y}k'_1(h)a_{21}\right]$\par
		nên $k'_2(0)=f(x_n,y_n)$.\par
		Do đó: $y'_n=b_1 k'_1(0)+b_2k'_2(0)$, nghĩa là $b_1+b_2=1$.\par
		Tiếp tục: $k''_1(h)=0.$
		\begin{align*}
			{k''}_2(h)&=2\left[\frac{\partial f}{\partial x}c_2+\frac{\partial f}{\partial y}{k'}_1(h)a_{21}\right]+h\left[\frac{\partial f}{\partial x}c_2+\frac{\partial f}{\partial y}{k'}_1(h)a_{21} \right]\\
			\Rightarrow {k''}_2(0)&=2\left[\frac{\partial f}{\partial x}c_2 +\frac{\partial f}{\partial y}{k'}_1(h)a_{21}\right]
		\end{align*}
		Do đó $\frac{\partial f_n}{\partial x}+\frac{\partial f_n}{\partial y}y'={y''}_n=2b_2\left(\frac{\partial f_n}{\partial x}c_2+\frac{\partial f_n}{\partial y}f_n a_{21}\right)$ cho nên ta có hệ phương trình:\par
		$$\begin{cases}
			b_1+b_2&=1\\
			b_2 c_2&=\frac12\\
			a_{21}b_2&=\frac12
		\end{cases}$$
		Hệ phương trình trên có vô số nghiệm.\par
		Với nghiệm $b_1=0$, $b_2=1$, $c_2=a_{21}=\frac12$, ta có công thức Runge – Kutta chính là công thức Euler cải tiến.\par
		Với nghiệm $b_1=b_2=\frac12$, $c_2=a_{21}=1$, ta có công thức Runge – Kutta 2 (RK2):\par
		$$\begin{cases}
			Y_2=y_n+hf(x_n,y_n)\\
			y_{n+1}=y_n+\frac12h\big[f(x_n,y_n)+f(x_n+h,Y_2)\big]
		\end{cases}$$
	\item Với $s=3$:\par
		Lập luận tương tự ta có hệ phương trình:\par
		$$\begin{cases}
			b_1+b_2+b_3=1\\
			b_2 c_2 + b_3 c_3 =\frac12\\
			b_2 c_2^2 + b_3 c_3^2=\frac13\\
			b_3 c_2 a_{32}=\frac16\\
		\end{cases}$$
		Một nghiệm của hệ phương trình thường dùng trong thực tế là:
		$$b_1=\frac16,~b_2=\frac23,~b_3=\frac16. c_2=a_{21}=\frac12,~c_3=1,a_{31}=-1.a_{32}=2$$
	\item Với $s=4$:\par
		Hệ phương trình với các ẩn số là hệ số của công thức Runge – Kutta 4 (RK4) là:\\
		$$\begin{cases}
			b_1+b_1+b_3+b_4=1\\
			b_2c_2+b_3c_3+b_4c_4=\frac12\\
			b_2c_2^2+b_3c_3^2+b_4c_4^2=\frac13\\
			b_2c_2^3+b_3c_3^3+b_4c_4^3=\frac14\\
			b_3a_{32}c_2+b_4a_{42}c_2+b_4a_{43}c_3=\frac16\\
			b_3c_3a_{32}c_2+b_4c_4a_{42}c_2+b_4c_4a_{43}c_3=\frac18\\
			b_3a_{32}c_2^2+b_4a_{42}c_2^2+b_4a_{43}c_3^2=\frac{1}{12}\\
			b_4a_{43}a_{32}c_2=\frac{1}{24}
		\end{cases}$$
		và
		$$\begin{cases}
			c_2=a_{21}\\
			c_3=a_{31}+a_{32}\\
			c_4=a_{41}+a_{42}+a_{43}
		\end{cases}$$
		Hệ phương trình trên có vô số nghiệm, trong thực tế RK4 thông dụng có dạng sau:
		$$\begin{cases}
			k_1=hf\left(x_n. y_n\right)\\
			k_2=hf\left(x_n+\frac h2,y_n+\frac{k_1}{2}\right)\\
			k_3=hf\left(x_n+\frac h2,y_n+\frac{k_2}{2}\right)\\
			k_4=hf\left(x_n+h,y_n+k_3\right)
			y_{n+1}=y_n+\frac16(k_1+2k_2+2k_3+k_4)
		\end{cases}$$
		Công thức RK4 có ước lượng sai số là : $R_4(h)=\frac{4\varphi^{(5)}\left(\xi\right)}{5!}h^5$.
\end{enumerate}

\subsection{Sơ đồ tính toán}
\begin{center}\begin{tabular}{|l|l|l|l|}
\hline
$n$ & $x_n$          & $y_n$                       & $hf(x_n,y_n)$ \\
\hline
$0$ & $x_0$          & $y_0$                       & ${}^{(0)}k_1$\\
    & $x_0+\frac h2$ & $y_0+\frac{{}^{(0)}k_1}{2}$ & ${}^{(0)}k_2$\\
    & $x_0+\frac h2$ & $y_0+\frac{{}^{(0)}k_2}{2}$ & ${}^{(0)}k_3$\\
    & $x_0+h$        & $y_0+{}^{(0)}k_3$           & ${}^{(0)}k_4$\\
\hline
\multicolumn{4}{|c|}{\tabularrowheight{24pt}$y_1=y_0+\frac16\left({}^{(0)}k_1+2{}^{(0)}k_2+2{}^{(0)}k_3+{}^{(0)}k_4\right)$}\\
\hline
$1$ & $x_1$          & $y_1$                       & ${}^{\left(1\right)}k_1$\\
\hline
\end{tabular}\end{center}
