\section{Một số khái niệm về phương trình vi phân thường}
Phương trình vi phân được nghiên cứu rộng rãi trong toán học thuần túy và ứng dụng, vật lý, các ngành kỹ thuật,\ldots và đóng một vai trò cực kỳ quan trọng. Song trong thực tế, việc tìm ra công thức của hàm, hay nghiệm chính xác của phương trình cụ thể gặp nhiều khó khăn, đồng thời, người ta cũng chỉ quan tâm tới giá trị của hàm số tại các giá trị cụ thể của các biến độc lập. Tuy nhiên, có nhiều bài toán mà việc tìm ra giá trị chính xác của hàm số tại một điểm nào đó thì không khả thi, khi đó, việc tìm được một giá trị xấp xỉ giá trị chính xác là phương án tối ưu.\par
Đó là mục tiêu của chương \textit{\textbf{Giải gần đúng phương trình vi phân thường}}.\par
% Ta tìm hiểu một số định nghĩa cơ bản sau:\par

\begin{definition}
	Phương trình vi phân thường cấp $n$ là phương trình có dạng:
	\begin{equation}\label{eqn:eqn601}
		F\left(x,y,y'(x), y''(x),\ldots, y^{(n)}(x) \right) = 0
	\end{equation}
	trong đó $x$ là biến số độc lập, $y=y(x)$ là hàm số phải tìm và $y'(x)$, $y''(x)$, \ldots, $y^{(n)}(x)$ là các đạo hàm của hàm số $y=y(x)$.
	\begin{itemize}
		\item \textbf{Cấp} của phương trình là cấp của đạo hàm cao nhất có mặt trong phương trình.
		\item \textbf{Nghiệm} của phương trình là mọi hàm số $y=y(x)$ thỏa mãn phương trình (\ref{eqn:eqn601}).
		\item \textbf{Giải} phương trình vi phân thường là tiến hành tìm tất cả các nghiệm của phương trình vi phân đó.
	\end{itemize}
\end{definition}

\begin{definition}
	Xét phương trình vi phân cấp $n$ có dạng:
	\begin{equation}\label{eqn:eqn602}
		y^{(n)}(x)=f\left(x,y(x),y'(x),~\ldots,~y^{(n-1)}(x)\right)
	\end{equation}
	Bài toán Cauchy đối với phương trình vi phân (\ref{eqn:eqn602}) là tìm hàm $y=y(x)$ thỏa mãn phương trình (\ref{eqn:eqn602}) và các điều kiện ban đầu:
	$$y(x_0)=y_0,~y'(x_0)=y'_0,~\ldots;~y^{(n-1)}(x_0)=y^{(n-1)}_0$$

	Bài toán Cauchy đối với phương trình vi phân cấp 1 là bài toán tìm nghiệm $y=y(x)$ của phương trình\\
	\begin{equation}\label{eqn:eqn603}
		y'=f(x,y)
	\end{equation}
	thỏa mãn điều kiện ban đầu
	\begin{equation}\label{eqn:eqn604}
		y(x_0)=y_0
	\end{equation}

	Phương trình vi phân (\ref{eqn:eqn603}) tương đương với phương trình tích phân:
	\begin{equation}\label{eqn:eqn605}
		y(x)= y_0 + \int\limits_{x_0}^{x} f\big(s, y\left(s\right)\big)\mathrm{d}s
	\end{equation}
	theo nghĩa mọi nghiệm của phương trình (\ref{eqn:eqn603}) là nghiệm liên tục của (\ref{eqn:eqn605}) và ngược lại.
\end{definition}
Báo cáo này sẽ trình bài ba phương pháp sau:\par
\begin{itemize}
	\item Phương pháp Euler và Euler cải tiến, gồm 3 phương pháp: Phương pháp Euler, Phương pháp Euler cải tiến thứ nhất và Phương pháp cải tiến thứ hai.
	\item Phương pháp Runge – Kutta.
	\item Phương pháp Adams, gồm 2 phương pháp: Công thức nội suy Adams và Công thức ngoại suy Adams.
\end{itemize}