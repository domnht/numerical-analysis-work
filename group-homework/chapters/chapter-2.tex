\chapter{Lý thuyết nội suy}
\textbf{Câu 1.} Tìm đa thức nội suy Larange của hàm số $y=f(x)$ cho bằng bảng sau:\par
\begin{multicols}{2}
d)\begin{center}\begin{tabular}{|c|c|c|c|c|}\hline
	$x$&321&322,8&324,2&325\\ \hline
	$y$&2,50651&2,50893&2,51081&2,51188\\ \hline
\end{tabular}\end{center}
và tính gần đúng giá trị $f(323,5)$.\par
e)\begin{center}\begin{tabular}{|c|c|c|c|c|c|}
	\hline
	$x$&-2&1&3&4&7\\ \hline
	$y$&12&37&51&67&127\\ \hline
\end{tabular}\end{center}
và tính gần đúng giá trị $f(5,1)$.\par
\end{multicols}

\textbf{Giải}\par
d) Ta có:
\begin{align*}
	y_0.L_0(x)	& =2,50651\cdot\frac{(x-322,8)(x-324,2)(x-325)}{(321-322,8)(321-324,2)(321-325)}\\
				& =\frac{250651}{100000}\cdot\frac{-25}{576}\left(x^3-\frac{4862}{5}x^2+\frac{7879661}{25}x-34053968\right)\\
				& =\frac{1420849532687}{384000}-\frac{1973417683019}{57600000}x+\frac{6767577}{64000}x^2-\frac{250651}{2304000}x^3
\end{align*}

Thay $x=323,5$, ta được $y_0.L_0(323,5)=-0,07996027$\par
\begin{align*}
	y_1.L_1(x)	& =2,50893\cdot\frac{(x-321)(x-324,2)(x-325)}{(322,8-321)(322,8-324,2)(322,8-325)}\\
				& =-\frac{188572098741}{12320}+\frac{6247598101}{44000}x-\frac{1756251}{4000}x^2+\frac{27877}{61600}x^3
\end{align*}

Thay $x=323,5$, ta được $y_1.L_1(323,5)=1,18794034$
\begin{align*}
	y_2.L_2(x)	& =2,51081\cdot\frac{(x-321)(x-322,8)(x-325)}{(324,2-321)(324,2-322,8)(324,2-325)}\\
				& =\frac{845543137491}{35840}-\frac{56108317827}{256000}x+\frac{43437013}{64000}x^2-\frac{251081}{358400}x^3
\end{align*}

Thay $x=323,5$, ta được $y_2.L_2(323,5)=1,83897216$
\begin{align*}
	y_3.L_3(x)	& =2,51188\cdot\frac{(x-321)(x-322,8)(x-324,2)}{(325-321)(325-322,8)(325-324,2)}\\
				& =-\frac{26369413998039}{2200000}+\frac{490348427793}{4400000}x-\frac{690767}{2000}x^2+\frac{62797}{176000}x^3
\end{align*}

Thay $x=323,5$, ta được $y_3.L_3(323,5)=-0,43708139$\par
Do đó ta có đa thức nội suy Larange có dạng: \begin{align*}
	P(x) & =y_0.L_0(x)+y_.L_1(x)+y_2.L_2(x)+y_3.L_3(x)\\
	     & =\frac{6766686623}{369600000}-\frac{47439221}{316800000}x+\frac{3}{6400}x^2-\frac{43}{88704000}x^3
\end{align*}
và \begin{align*}
	L(323,5) &=y_0.L_0(323,5)+y_.L_1(323,5)+y_2.L_2(323,5)+y_3.L_3(323,5)\\
	         &=2,50987084
\end{align*}
Vậy giá trị gần đúng của $f(323,5)$ là $P(323,5)\approx 2,50987084$\par

e) Ta có:
\begin{align*}
	y_0L_0(x)	& =12\cdot\frac{(x-1)(x-3)(x-4)(x-7)}{(-2-1)(-2-3)(-2-4)(-2-7)}\\
				& =\frac{56}{45}-\frac{58}{27}x+\frac{10}{9}x^2-\frac{2}{9}x^3+\frac{2}{135}x^4
\end{align*}
\begin{align*}
	y_1L_1(x)	& =37\cdot\frac{(x+2)(x-3)(x-4)(x-7)}{(1+2)(1-3)(1-4)(1-7)}\\
				& =\frac{518}{9}-\frac{703}{54}x-\frac{407}{36}x^2+\frac{37}{9}x^3-\frac{37}{108}x^4
\end{align*}
\begin{align*}
	y_2L_2(x)	& =51\cdot\frac{(x+2)(x-1)(x-4)(x-7)}{(3+2)(3-1)(3-4)(3-7)}\\
				& =-\frac{357}{5}+\frac{255}{4}x+\frac{153}{8}x^2-\frac{51}{4}x^3+\frac{51}{40}x^4
\end{align*}
\begin{align*}
	y_3L_3(x)	&=67\cdot\frac{(x+2)(x-1)(x-3)(x-7)}{(4+2)(4-1)(4-3)(4-7)}\\
				&=\frac{469}{9}-\frac{2747}{54}x-\frac{67}{6}x^2+\frac{67}{6}x^3-\frac{67}{54}x^4
\end{align*}
\begin{align*}
	y_4L_4(x)	& =127\cdot\frac{(x+2)(x-1)(x-3)(x-4)}{(7+2)(7-1)(7-3)(7-4)}\\
				& =-\frac{127}{27}+\frac{1651}{324}x+\frac{127}{216}x^2-\frac{127}{108}x^3+\frac{127}{648}x^4
\end{align*}

Do đó ta có đa thức nội suy Larange có dạng:
\begin{align*}
	P(x)	& =y_0.L_0(x)+y_.L_1(x)+y_2.L_2(x)+y_3.L_3(x)\\
			& =\frac{4699}{135} + \frac{455}{162}x - \frac{89}{54}x^2 + \frac{61}{54}x^3 - \frac{79}{810}x^4
\end{align*}

và giá trị gần đúng của $f(5,1)$ là $P(5,1)\approx 90,1281$.

\textbf{Câu 2b.} Tìm đa thức nội suy Newton của hàm số $y=f(x)$ được cho bằng bảng sau:\par
\begin{longtable}{|c|c|c|c|c|c|}\hline
	$x$ & $-0,35$ & $-0,1$ & $0,15$ & $0,4$ & $0,65$\\ \hline
	$y$ & $0,387322$ & $0,762616$ & $1,501553$ & $2,956482$ & $5,821162$\\ \hline
\end{longtable}
và tính gần đúng giá trị $f\left(0,55\right)$.\par

\textbf{Giải}\par

Ta có bảng tỉ sai phân như sau:
\begin{longtable}{|c|c|c|c|c|c|}\hline
	$x_i$ & $y_i$ & TSP cấp 1 & TSP cấp 2 & TSP cấp 3 & TSP cấp 4\\ \hline
	\endhead
	$-0,35$ & $0,387322$ &&&& \\ \hline
	&&$1,501176$&&& \\ \hline
	$-0,1$ & $0,762616$&&$2,909144$&& \\ \hline
	&&$2,955748$&&$3,758389$&\\ \hline
	$0,15$&$1,501553$&&$5,727936$&&$3,641707$\\ \hline
	&&$5,819716$&&$7,400096$& \\ \hline
	$0,4$&$2,956482$&&$11,278008$&& \\ \hline
	&&$11,45872$&&& \\ \hline
	$0,65$&$5,821162$&&&& \\ \hline
\end{longtable}

Đa thức nội suy Newton:
\begin{align*}
	f(x) =~& 0,387323 + 1,501176(x+0,35) + 2,909144(x+035)(x+0,1) +\\
	&+ 3,758389(x+0,35)(x+0,1)(x-0,15) +\\
	&+ 3,641707(x+0,35)(x+0,1)(x-0,15)(x-0,4)
\end{align*}

Khi đó, ta tính được $f(0,55)=4,447517528$.\par

\textbf{Câu 3b.} Tìm đa thức nội suy Newton tiến và lùi của hàm số $y=f(x)$ được cho bằng bảng sau:\par
\begin{longtable}{|c|c|c|c|c|c|}\hline
	$x$ & 1,9 & 2,1 & 2,3 & 2,5 & 2,7 \\ \hline
	$y$ & 11,18 & 14,78 & 17,89 & 23,52 & 28,56 \\ \hline
\end{longtable}
và tính gần đúng giá trị $f(2,37)$.\par

\textbf{Giải}\par
Ta có bảng sai phân:
\begin{longtable}{|c|c|c|c|c|c|}\hline
	$x_i$&$y_i$&$\Delta y$& $\Delta^2 y$&$\Delta^3 y$&$\Delta^4 y$\\\hline
	\endhead
	1,9&11,18&& &&\\\hline
	&&3,6& &&\\\hline
	2,1&14,78&& -0,49&&\\\hline
	&&3,11& &3,01&\\\hline
	2,3&17,89&& 2,52&&-6,12\\\hline
	&&5,63& &-3,11&\\\hline
	2,5&23,52&& -0,59&&\\\hline
	&&5,04& &&\\\hline
	2,7&28,56&& &&\\\hline
\end{longtable}
Do đó:\par

\textit{Đa thức nội suy Newton dạng tiến} là:
\begin{multline*}
	P_t(x)=11,18 + 18(x-1,9) -\frac{49}{8}(x-1,9)(x-2,1) + \frac{1505}{24}(x-1,9)(x-2,1)(x-2,3)\\
	-\frac{1275}{8}(x-1,9)(x-2,1)(x-2,3)(x-2,3)(x-2,5)
\end{multline*}
và \textit{Đa thức nội suy Newton dạng lùi} là:
\begin{multline*}
	P_{l} (x)= 28,56 +25,2(x-2,7) -\frac{59}{8}(x-2,7)(x-2,5)-\frac{1555}{24}(x-2,7)(x-2,5)(x-2,3) \\
	-\frac{1275}{8}(x-2,7)(x-2,5)(x-2,3)(x-2,1)
\end{multline*}
Khi đó ta tính được các giá trị:
\begin{center}
	$f_t(2,37)\approx P_t(2,37)= 19,60382028$ và $f_l(2,37) \approx P_l(2,37)= 19,63382028$
\end{center}

\textbf{Câu 5b.} Tính gần đúng tổng sau:
$$S_n=1^2 + 4^2 +7^2 + \ldots + (3n+1)^2 $$
biết $S_n$ là một đa thức bậc ba.\par

\textbf{Giải}\par
Ta có bảng sau:
\begin{longtable}{|c|c|c|c|c|}\hline
	$n$&$S_n$&$\Delta y$&$\Delta^2 y$&$\Delta^3 y$\\\hline
	\endhead
	0&1&&&\\\hline
	&&16&&\\\hline
	1&17&&33&\\\hline
	&&49&&18\\\hline
	2&66&&51&\\\hline
	&&100&&\\\hline
	3&166&&&\\\hline
\end{longtable}

Áp dụng công thức nội suy Newton dạng tiến ta tìm được:
$$S_n= 1+ 16n + \frac{33}{2}n(n-1)+3n(n-1)(n-2)$$

\textbf{Câu 8c.} Cho bảng giá trị:\\
\begin{longtable}{|c|c|c|c|c|c|c|}\hline
	$x$&2&4&6&8&10&12\\\hline
	$f(x_i)$&7,32&8,24&9,2&10,19&11,01&12,05\\\hline
\end{longtable}

Tìm hàm xấp xỉ bằng phương pháp bình phương bé nhất biết quan hệ giữa $x$ và $y=f(x)$ là:
$$y=ax^b.$$

\textbf{Giải}\par
Ta có $y=ax^b\Leftrightarrow\ln y=\ln\left(ax^b\right)=\ln a+b\ln x$.\par
Đặt $Y=\ln y$, $X=lnx$, $A=b$, $B=\ln a$, ta có $Y=AX+B$.\par
Ta lập bảng:
\begin{longtable}{|Sc|c|c|c|c|c|c|}\hline
	&$x_i$&$y_i$&$X_i=\ln{x_i}$&$Y_i=\ln{y_i}$&$X^2$&$XY$\\\hline
	\endhead
	&2&7,32&0,693147&1,99061&0,480453&1,379785\\\hline
	&4&8,24&1,386294&2,109&1,921811&2,923694\\\hline
	&6&9,2&1,791759&2,219203&3,2104&3,976277\\\hline
	&8&10,19&2,079442&2,321407&4,324079&4,827231\\\hline
	&10&11,01&2,302585&2,398804&5,301898&5,52345\\\hline
	&12&12,05&2,484907&2,489065&6,174763&6,185095\\\hline
	$\sum$&42&58,01&10,738134&13,528089&21,413404&24,815532\\\hline
\end{longtable}
Vậy hệ số $A,B$ được xác định bởi hệ phương trình sau:
$$\begin{cases}
10,738134A+6B=13,528089 \\
21,413404A+10,738134B=24,815532
\end{cases}$$

Giải hệ phương trình ta nhận được: $\begin{cases}
	A=0,275320\\ 
	B=1,761945\\ 
\end{cases}$, từ đó: $\begin{cases}
	a=\mathrm{e}^{B}=5,823754 \\
	b=A=0,275320
\end{cases}$.\par
Vậy $y=5,823754 x^{0,275320}$.\par

\textbf{Câu 9a.} Cho bảng giá trị:
\begin{longtable}{|c|c|c|c|c|c|c|}\hline
	$x_i$&19&22&25&28&32&35\\\hline
	$f(x_i)$&0,66&0,367&0,223&0,14&0,084&0,06\\\hline
\end{longtable}
Tìm hàm xấp xỉ bằng phương pháp bình phương bé nhất biết quan hệ giữa $x$ và $y=f(x)$ là: $$y=ax^2+bx+c$$

\textbf{Giải}\par
Ta có bảng số liệu sau:
\begin{longtable}{|Sc|c|c|c|c|c|c|c|}\hline
	&$x_i$&$y_i$&$x_i^2$&$x_iy_i$&$x_i^3$&$x_i^4$&$x_i^2y_i$\\\hline
	\endhead
	&19&0,66&361&12,54&6859&130321&238,26\\\hline
	&22&0,367&484&8,074&10648&234256&177,628\\\hline
	&25&0,223&625&5,575&15625&390625&139,375\\\hline
	&28&0,14&784&3,92&21952&614656&109,76\\\hline
	&32&0,084&1024&2,688&32768&1048576&86,016\\\hline
	&35&0,06&1225&2,1&42875&1500625&73,5\\\hline
	$\sum$&161&1,534&4503&34,897&130727&3919059&824,539\\\hline
\end{longtable}

$a,b,c$ thỏa mãn hệ phương trình:
$$\begin{cases}
	3919059a + 130727b + 4503c &=  824,539 \\ 
	130727a + 4503b + 161c &=  3 4,897 \\ 
	4503a+161b+6c&=1,534
\end{cases}
\Leftrightarrow
\begin{cases}
	a=&3,202\cdot 10^{-3}\\ 
	b=& -0,207577\\
	c=&3,422690
\end{cases}$$
Vậy $f(x)=3,{{202\cdot 10}^{-3}}{{x}^{2}}-0,207577x+3,422690.$
\par
\textbf{Câu 10e.} Cho bảng giá trị:
\begin{longtable}{|c|c|c|c|c|c|c|c|c|}\hline
	$x_i$&0&1&2&3&4&5&6&7\\ \hline
	$f(x_i)$&1,4&1,3&1,4&1,1&1,3&1,8&1,6&2,3\\ \hline
\end{longtable}
Tìm hàm xấp xỉ bằng phương pháp bình phương bé nhất biết quan hệ giữa $x$ và $y=f(x)$ là: $$y=\mathrm{e}^{ax+b}$$

\textbf{Giải}\par

Ta có $y={{e}^{ax+b}}\Leftrightarrow \ln y=ax+b$.\par
Đặt $Y=\ln y$ ta được hàm $Y=ax+b$.\par
Ta có bảng số liệu sau:
\begin{longtable}{|Sc|c|c|c|c|c|}\hline
	$x_i$&$y_i$&$Y_i=\ln{y_i}$&$x_i^2$&$x_iY_i$\\\hline
	\endhead
	0&1,4&0,336472&0&0\\\hline
	1&1,3&0,262364&1&0,262364\\\hline
	2&1,4&0,336472&4&0,672944\\\hline
	3&1,1&0,09531&9&0,28593\\\hline
	4&1,3&0,262364&16&1,049456\\\hline
	5&1,8&0,587787&25&2,938935\\\hline
	6&1,6&0,470004&36&2,820024\\\hline
	7&2,3&0,832909&49&5,830363\\\hline
	$\sum$28&12,2&3,183682&140&13,860016\\\hline
\end{longtable}

Từ đó ta có hệ phương trình:\par
$\begin{cases}
	140a+28b&=13,86002\\
	28a+8b&=3,183683
\end{cases} \Leftrightarrow \begin{cases}
	a& =0,064694 \\ 
	b& =0,171533
\end{cases}.$\par

Vậy $Y=0,064694x+0,171533.$\par

Từ đó ta có $y=\mathrm{e}^{0,064694x+0,171533}$.
