\chapter{Giải gần đúng phương trình đại số và siêu việt}

\textbf{Bài 2.} Dùng phương pháp lặp đơn, hãy tìm nghiệm của các phương trình:\par
c) $x-\sin{x}=0,25$ với sai số $10^{-2}$ trong khoảng phân lý nghiệm $(1;1,5)$.\par
f) $2^{x}-5x-3=0$ với sai số $10^{-4}$ trong khoảng phân ly nghiệm $(4;5)$.\par
i) $(x-1)^2=\frac{1}{2}e^x$ với sai số $10^{-2}$ trong khoảng phân ly nghiệm $(0;0,5)$.\par
j) $x=\ln x+3$ với sai số $10^{-3}$ trong khoảng phân ly nghiệm $(4;5)$

\textbf{Giải}

c) Đặt $f(x)=x-\sin{x}-0,25 $. Ta có:
\begin{itemize}
	\item $f(x) $ liên tục trên khoảng $\left(1;1,5\right) $.
	\item $f'(x)=1-\cos{x}>0$, $\forall x\in \left(1;1,5\right) $ nên hàm số đồng biến trên đoạn $\left(1;1,5\right)$.
	\item $f\left(1\right)= 0,7325$, $f\left(1,5\right)=1,2238 $, suy ra $f\left(1\right)f\left(1,5\right)>0$.
\end{itemize}
Từ đây ta suy ra hàm số vô nghiệm trên đoạn $(1;1,5)$.\par

f) Đặt $f(x)=2^{x}-5x-3 $. Ta có:
\begin{itemize}
	\item $f(x) $ liên tục trên khoảng $\left(4;5\right) $.
	\item $f\left(4\right)f\left(5\right)<0 $.
	\item $f'(x)=2^{x}\ln{2}-5 > 0, \forall x \in \left(4;5\right) $.
\end{itemize}

Do đó: phương trình $f(x)=0  $ có một nghiệm trên khoảng $\left(4;5\right) $.\par

Do $f'(x)>0 $ nên ta đặt $\varphi(x)=x-\frac{f(x)}{M}$.
Trong đó:
$$M\geq \underset{x\in \left(4;5\right) }\max\left| f'(x) \right| \approx 17,1807$$

Chọn $M=17,1807 $, suy ra $\varphi(x) =x-\frac{f(x) }{17,1807} = \frac{-2^{x}+22,1807x + 3}{17,1807}$.\par
Ta có $\varphi ' (x) = \frac{-2^{x}\ln{2}+22,1807}{17,1807}$ và $\max_{x\in(4;5)}\left| \varphi ' (x) \right| < \left| \varphi ' \left(4\right) \right| = 0,6455 $.\par

Chọn $L=0,6455 $.\par
Chọn $x_0=4,7 $, ta có xấp xỉ nghiệm trong bảng sau:
\begin{longtable}{|c|c|c|}\hline
	$n$ & $x_n=\varphi\left(x_{n-1}\right) $ & $\left| x_n - x* \right| \leq 1,82087 \left|x_n - x_{n-1}\right|$\\ \hline
	\endhead
	$1$ & $4,72956$ & $0,05382$\\ \hline
	$2$ & $4,73641$ & $0,01247$ \\ \hline
	$3$ & $4,73791$ & $0,02731$ \\ \hline
	$4$ & $4,73822$ & $0,00056$ \\ \hline
	$5$ & $4,73829$ & $0,00013$ \\ \hline
	$6$ & $4,73831$ & $0,00004$ \\ \hline
\end{longtable}
Vậy $x^*\approx x_6=4,73831$.\par

i) Đặt $f(x)=(x-1)^2-\frac{1}{2}e^x=0$, ta có:
\begin{align*}
	&f'(x) =2(x-1)-\frac{1}{2}e^x<0 \forall x\in (0;0,5)\\
	&f(0)  =\frac{1}{2}\\
	&f(0,5)=\frac{1}{4}-\frac{1}{2}.e^{\frac{1}{2}}
\end{align*}

Từ đây ta có $f(0).f(0,5)<0$ và $f(x)$ đơn điệu giảm trên khoảng $(0;0,5)$ nên phương trình $f(x)=0$ có duy nhất nghiệm trên khoảng $(0;0,5)$\par

Phương trình đã cho tương đương với $$x=1-\sqrt{\frac{e^x}{2}}$$
Đặt $\varphi (x)=1-\sqrt{\frac{e^x}{2}}$, ta có:
\begin{align*}
	&\varphi'(x)=-\sqrt{\frac{e^x}{8}}\\
	&\max_{x\in [0;0,5]} |\varphi'(x)|\approx 0,45397
\end{align*}
Do đó $|x_n-x^*|\leqslant 0,83140|x_n-x_{n-1}|$
Chọn $x=0,1$, ta có xấp xỉ nghiệm của phương trình được cho trong bảng sau:

\begin{longtable}{|c|c|c|}\hline
$n$&$x_n=\varphi(x_{n-1})$&$0,83140|x_n-x_{n-1}|$\\ \hline
\endhead
1&$0,25664$&$0,13023$\\ \hline
2&$0,19608$&$0,05035$\\ \hline
3&$0,22006$&$0,01994$\\ \hline
2&$0,21065$&$0,78210.10^{-2}$\\ \hline
\end{longtable}

Vậy nghiệm của phương trình đã cho với sai số $10^{-2}$ trong khoảng phân ly nghiệm $(0;0,5)$ là $x\approx 0,21065$.\par

j) Đặt $f(x)=x-\ln x-3$, ta có:
\begin{align*}
	&f'(x)=1-\frac{1}{x}>0~\forall x\in (4;5)\\
	&f''(x)=-\frac{1}{x^2}\\
	&f(4)=1-\ln 4\\
	&f(5)=2-\ln 5
\end{align*}

Từ đây ta có $f(4).f(5)<0$ và $f(x)$ đơn điệu tăng trên khoảng $(4;5)$ nên phương trình $f(x)=0$ có duy nhất nghiệm trên khoảng $(4;5)$.\par

Đặt $\varphi(x)=\ln x+3$, ta có:
$$\varphi'(x)=\frac{1}{x};~~~~
\max_{x\in [4;5]} |\varphi'(x)| = 0,25$$

Do đó $|x_n-x^*|\leqslant \frac{1}{3}|x_n-x_{n-1}|$\par
Chọn $x=4,1$, ta có xấp xỉ nghiệm của phương trình được cho trong bảng sau:
\begin{longtable}{|c|c|Sc|}\hline
$n$&$x_n=\varphi(x_{n-1})$&$\frac{1}{3}|x_n-x_{n-1}|$\\ \hline
1&$4,41099$&$0,10366$\\ \hline
2&$4,48410$&$0,02437$\\ \hline
3&$4,50054$&$0,54797.10^{-2}$\\ \hline
4&$4,50420$&$0,12198.10^{-2}$\\ \hline
5&$4,50500$&$0,27092.10^{-3}$\\ \hline
\end{longtable}

Vậy nghiệm của phương trình đã cho với sai số $10^{-3}$ trong khoảng phân ly nghiệm $(4;5)$ là $x\approx 4,50500$.\par
