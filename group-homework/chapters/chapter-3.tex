\chapter{Tính gần đúng đạo hàm và tích phân}

\textbf{Bài 3.} Bằng phương pháp hình thang và Simpson 1/3, với $n=10$, tính gần đúng và đánh giá sai số các tích phân sau:\par
\begin{multicols}{2}
  b) $I=\int\limits_{0}^{\pi} \sin x\mathrm{d}x$\par
  d) $I=\int\limits_{0}^{6} \frac{1}{x^2+1}\mathrm{d}x$\par
  f) $I= \int\limits^{4}_{2} \frac{1}{\left(x-1\right)^2}\mathrm{d}x$\par
  j) $I=\int\limits_{0,1}^{1,1} \frac{1}{\left(1+4x\right)^2}\mathrm{d}x$\par
\end{multicols}

\textbf{Giải}\par

b)~$\star$~\textit{Công thức Simpson 1/3}\par

$h=\frac{\pi-0}{10}=\frac{\pi}{10}$. Ta có bảng sau:
\begin{longtable}{|c|Sc|c|c|c|}\hline
$i$&$x_i$&\multicolumn{3}{c|} {$y_i=f(x_i)= \sin x$}\\ \hline
0&0&0& & \\ \hline
\endhead
1&$\frac{\pi}{10}$& &0,3090& \\ \hline
2&$\frac{\pi}{5}$& & &0,5878 \\ \hline
3&$\frac{3\pi}{10}$& &0,8090& \\ \hline
4&$\frac{2\pi}{5}$& & &0,9511 \\ \hline
5&$\frac{\pi}{2}$& &1& \\ \hline
6&$\frac{3\pi}{5}$& & &0,9511 \\ \hline
7&$\frac{7\pi}{10}$& &0,8090& \\ \hline
8&$\frac{4\pi}{5}$& & &0,5878 \\ \hline
9&$\frac{9\pi}{10}$& & 0,3090& \\ \hline
10&$\pi$&0& & \\ \hline
\end{longtable}

Theo công thức Simpson 1/3, ta có:
$$I_S\approx \frac{\pi}{30}.\left[0+4.3,2361+2.3,0777\right]\approx 2,000105435$$

Đánh giá sai số:
$$\lvert I-I_S \rvert\leqslant\frac{\displaystyle\max_{0\leqslant x\leqslant\pi}\left\lvert f^{(4)}(x)\right\rvert}{180}(\pi -0)h^4=\frac{1}{180}.\pi.\left(\frac{\pi}{10}\right)^4\approx 0,00017$$

$\star$~\textit{Công thức hình thang}\par
Ta có $h=\frac{\pi-0}{10}=\frac{\pi}{10}$\par

Ta được bảng sau:
\begin{center}
\begin{tabular}{|Sc|c|c|}\hline
$x_i$&\multicolumn{2}{c|} {$y_i=f(x_i)= \sin x$}\\ \hline
0&	0&	\\ \hline
$\frac{\pi}{10}$&		&0,3090\\ \hline
$\frac{\pi}{5}$&		&0,5878\\ \hline
$\frac{3\pi}{10}$&		&0,8090\\ \hline
$\frac{2\pi}{5}$&		&0,9511\\ \hline
$\frac{\pi}{2}$&		&1\\ \hline
$\frac{3\pi}{5}$&		&0,9511\\ \hline
$\frac{7\pi}{10}$&		&0,8090\\ \hline
$\frac{4\pi}{5}$&		&0,5878\\ \hline
$\frac{9\pi}{10}$&		&0,3090\\ \hline
1&	0&	\\ \hline
&0&6,3138 \\ \hline
\end{tabular}
\end{center}

Do đó giá trị gần đúng của tích phân đã cho là:
$$I_T\approx \frac{\pi}{2.10}.(0+2.6,3138)\approx 1,9835$$

* Đánh giá sai số:
Ta có $M=\max_{0\leqslant x\leqslant\pi}\lvert f''(x)\rvert=1$ và $\bar{I}=1,98$\par
nên $\lvert I_T -\bar{I}\rvert\leqslant\frac{M}{12}.(\pi -0).\left(\frac{\pi}{10}\right)^2\approx 0,026$\par
và $\lvert I_T -\bar{I}\rvert=3,5.10^{-3}$\par
Do đó $\lvert I-\bar{I}\rvert\leqslant\lvert I-I_T\rvert+\lvert I_T-\bar{I}\rvert\leqslant 0,0295$.\par

d)~$\star$~\textit{Công thức Simpson 1/3}\par
Ta có $h=\frac{6-0}{10}=\frac35$\par
Ta có bảng sau:
\begin{longtable}{|c|c|c|c|c|}\hline
$i$&$x_i$&\multicolumn {3} {Sc|} {$y_i=f(x_i)= \frac{1}{x^2+1}$}\\ \hline
0&0&1& & \\ \hline
1&$0,6$& &0,7353& \\ \hline
2&$1,2$& & &0,4098 \\ \hline
3&$1,8$& &0,2358& \\ \hline
4&$2,4$& & &0,1479 \\ \hline
5&$3$& &$0,1$& \\ \hline
6&$3,6$& & &0,0716 \\ \hline
7&$4,2$& &0,0536& \\ \hline
8&$4,8$& & &0,0416 \\ \hline
9&$5,4$& & 0,0332& \\ \hline
10&$6$&0,0270& & \\ \hline
\end{longtable}

Theo công thức Simpson 1/3, ta có:
$$I_S\approx \frac{1}{5}.\left[1,0270+4.1,1579+2.0,6980\right]\approx 1,410973$$

Đánh giá sai số:
$$\lvert I-I_S \rvert\leqslant\frac{\max_{0\leqslant x\leqslant 6}\left\lvert f^{(4)}(x)\right\rvert}{180}(6 -0)h^4=\frac{24}{180}.6.\left( \frac{3}{5}\right)^4= 0,10368  $$

$\star$~\textit{Công thức hình thang}

Ta có $h=\frac{6-0}{10}=\frac{3}{5}$
Ta có bảng sau:
\begin{longtable}{|c|c|Sc|Sc|}\hline
$i$&$x_i$&\multicolumn{2}{Sc|} {$y_i=f(x_i)= \frac{1}{x^2+1}$}\\ \hline
\endhead
0&0&1&  \\ \hline
1&$0,6$& &$\frac{25}{34}$ \\ \hline
2&$1,2$&  &$\frac{25}{61}$ \\ \hline
3&$1,8$& &$\frac{25}{106}$ \\ \hline
4&$2,4$&  &$\frac{25}{169}$ \\ \hline
5&$3$& &$\frac{1}{10}$ \\ \hline
6&$3,6$&  &$\frac{25}{349}$ \\ \hline
7&$4,2$& &$\frac{25}{466}$ \\ \hline
8&$4,8$&  &$\frac{25}{601}$ \\ \hline
9&$5,4$& &$\frac{25}{754}$ \\ \hline
10&$6$&$\frac{1}{37}$& \\ \hline
& &$\frac{38}{37}$&$\frac{11967477}{6543383}$\\ \hline
\end{longtable}

Do đó giá trị gần đúng của tích phân đã cho là:
$$I_T\approx \frac{6-0}{2.10}.(\frac{38}{37}+2.\frac{11967477}{6543383})=1,40547$$

*~Đánh giá sai số
Ta có $M=\max_{0\leqslant 6}\lvert f''(x)\rvert= 2$ và $\bar{I}=1,41$\par
nên $\lvert I_T -\bar{I}\rvert\leqslant\frac{M}{12}.(6 -0).\left(\frac{3}{5}\right)^2=0,36$\par
và $\lvert I_T -\bar{I}\rvert=4,53.10^{-3}$\par
Do đó $\lvert I-\bar{I}\rvert\leqslant\lvert I-I_T\rvert+\lvert I_T-\bar{I}\rvert\leqslant 0,364653$.\par

f) Ta có $h=\frac{b-a}{n}=\frac{4-2}{10}=0,2$ và $f(x)=\frac{1}{\left(x-1\right)^2}$\par

$\star$~\textit{Công thức hình thang}\par
Ta có bảng sau:
\begin{longtable}{|c|Sc|Sc|}\hline
  $x_i$ & \multicolumn{2}{Sc|}{$y_i=f\left(x_i\right)=\frac{1}{\left(x_i-1\right)^2}$}\\ \hline
  \endhead
  $2,0$ & $1$ & \\ \hline
  $2,2$ && $\frac{25}{36}$\\ \hline
  $2,4$ && $\frac{25}{49}$\\ \hline
  $2,6$ && $\frac{25}{64}$\\ \hline
  $2,6$ && $\frac{25}{81}$\\ \hline
  $3,0$ && $  \frac{1}{4}$\\ \hline
  $3,2$ && $\frac{25}{121}$\\ \hline
  $3,4$ && $\frac{25}{144}$\\ \hline
  $3,6$ && $\frac{25}{169}$\\ \hline
  $3,8$ && $\frac{25}{196}$\\ \hline
  $4,0$ & $\frac19$ & \\ \hline
        & $\frac{10}{9}$ & $2,809618197$\\ \hline
\end{longtable}

Vậy theo công thức hình thang ta tính được giá trị gần đúng của tích phân là:
$$I\approx \int\limits^{4}_{2} \frac{1}{\left(x-1\right)^2}\mathrm{d}x=\frac{4-2}{2\left(10\right)}\left(\frac{10}{9}+2\left(2,809618197\right) \right) = 0,6730347505$$

Nếu làm tròn đến năm chữ số thập phân thì $I_T = 0,67303$.\par

Đánh giá sai số theo công thức tích phân, ta có:
$$f'(x)=-\frac{2}{\left(x-1\right)^3};~~~~f''(x)=\frac{6}{\left(x-1\right)^4}$$
%underset
Do hàm $f''$ nghịch biến trên đoạn $[2;4]$ nên $M=\max_{2\leqslant x\leqslant 4}\left|f''(x)\right|=\left|f''\left(2\right)\right|=6$.\par

Nên $\left|I-I_T\right| \leqslant\frac{6}{12} \left(4-2\right)\left(0,2\right)^2= 0,04$.\par

Và $\left|I_T-\overline{I} \right| = 4,7505.10^{-6}$\par

Do đó $\left| I - \overline{I} \right| \leqslant\left|I - I_T\right| + \left|I_T - \overline{I}\right| = 0,04 + 4,7505.10^{-6}$.\par

$\star$~\textit{Áp dụng công thức Simpson 1/3}\par
Ta có bảng:
\begin{longtable}{|c|c|Sc|Sc|Sc|}\hline
  $i$ & $x_i$ & \multicolumn{3}{Sc|}{$f\left(x_i\right)=\frac{1}{\left(x_i - 1 \right)^2}$}\\ \hline
  \endhead
  $0$ & $2,0$ & $1$ & &\\ \hline
  $1$ & $2,2$ & & $\frac{25}{36}$ &\\ \hline
  $2$ & $2,4$ & & & $\frac{25}{49}$ \\ \hline
  $3$ & $2,6$ & & $\frac{25}{64}$ & \\ \hline
  $4$ & $2,8$ & & & $\frac{25}{81}$ \\ \hline
  $5$ & $3,0$ & & $\frac{1}{4}$ & \\ \hline
  $6$ & $3,2$ & & & $\frac{25}{121}$ \\ \hline
  $7$ & $3,4$ & & $\frac{25}{144}$ &\\ \hline
  $8$ & $3,6$ & & & $\frac{25}{169}$ \\ \hline
  $9$ & $3,8$ & & $\frac{25}{196}$ & \\ \hline
  $10$ & $4,0$ & $\frac{1}{9}$ & & \\ \hline
  & & $\frac{10}{9}$ & $1,636231576$ & $1,173386621$ \\ \hline
\end{longtable}

Áp dụng công thức Simpson 1/3 ta tính gần đúng tích phân là:
$$I_S=\int\limits^{4}_{2} \frac{1}{\left(x-1\right)^2}\mathrm{d}x \approx \frac{0,2}{3} \left[ \frac{10}{9} + 4\left(1,6366231576\right)+ 2\left(1,173386621\right) \right]=0,6668540438$$

Nếu lấy 5 chữ số thập phân, khi đó $\overline{I}=0,66685$. Nên $\left| I_S - \overline{I} \right| = 4.0438\times 10^{-6}$\par

Đánh giá sai số theo công thức, ta có:
$$f^{\left(3\right)}(x) =\frac{24}{\left(x-1\right)^5};~~~~f^{\left(4\right)}(x)=\frac{120}{\left(x-1\right)^6}$$

Do $f^{\left(4\right)}(x)$ là hàm nghịch biến trên đoạn $\left[2;4\right]$ nên $M=\underset{2 \leqslant x \leqslant 4}{\max} \left| f^{\left(4\right)}(x) \right| = \left| f^{\left(4\right)}\left(2\right) \right|= 120$\par

Do đó $\left| I - I_{S} \right| \leqslant\frac{120}{180}\times \left(4-2\right) \times \left(0,2\right)^4=0,05\left(3\right) $\par

Vậy $\left| I-\overline{I}\right| \leqslant\left| I- I_S\right| + \left|I_S -\overline{I}\right|=0,05\left(3\right) - 4.0438\times 10^{-6}$\par

\textbf{Bài 5.} Tính gần đúng tích phân $I=\int\limits_{-0,8}^{0,8} \frac{\sin^2 x}{\sqrt{1-\cos x}}\mathrm{d}x$ bằng công thức Simpson với $n=16$ và đánh giá sai số của kết quả vừa nhận được.\par

\textbf{Giải}\par
Ta có $h=\frac{0,8-(-0,8)}{16}=0,1$.\par
Ta lập được bảng sau:
\begin{longtable}{|c|c|c|c|c|}\hline
$i$&   $x_i$&\multicolumn {3} {Sc|} {$y_i=f(x_i)= \frac{\sin^2 x}{\sqrt{1-\cos x}}$}\\ \hline
\endhead
  0&	$-0,8$&	0,934411509&			&		\\ \hline
  1&	$-0,7$&			   &0,85582621	&		\\ \hline
  2&	$-0,6$&			   &			&0,762860112 	\\ \hline
  3&	$-0,5$&			   &0,656932407 &		    \\ \hline
  4&	$-0,4$&	 	       &				&0,539742953	           \\ \hline
  5&	$-0,3$&		       &0,413235796 &                          \\ \hline
  6&	$-0,2$&		       &			&0,279557228	           \\ \hline
  7&	$-0,1$&		       &0,141009326 &                 \\ \hline
  8&	$0,001$&		       &			&0,001414213	           \\ \hline
  9&	$0,1$&		       &0,141009326 &                   \\ \hline
 10&	$0,2$&		       &			&0,279557228	           \\ \hline
 11&	$0,3$&		       &0,413235796 &                    \\ \hline
 12&	$0,4$&		       &			&0,539742953	              \\ \hline
 13&	$0,5$&		       &0,656932407	&					\\ \hline
 14&	$0,6$&			   &            &0,762860112		\\ \hline
 15&	$0,7$&		       &0,85582621	&		\\ \hline
 16&	$0,8$&	0,934411509&		    &		\\ \hline
   &	   &     1,868823017&4,134007477&	3,165734799	\\ \hline
\end{longtable}

Theo công thức Simpson 1/3, ta có:
$$I_S\approx \frac{0,1}{3}\cdot\left[1,868823017+4.4,134007477 +2.3,165734799\right]\approx 0,824544084$$

*~Đánh giá sai số:
$$\lvert I-I_S \rvert\leqslant\frac{\displaystyle\max_{-0,8\leqslant x\leqslant 0,8}\left\lvert f^{(4)}(x)\right\rvert}{180}(0,8+0,8)h^4=\frac{3,35366}{180}.1,6.\left( 0,1\right)^4\approx1,98103.10^{-6}$$

j) Ta có $h=\frac{1,1-0,1}{10}=0,1$ và $g(x)=\frac{1}{\left(1+4x\right)^2}$\par

Ta tìm được các đạo hàm của $g(x)$:
$$g'(x) = -\frac{8}{\left(1+4x\right)^3};~~~~
g''(x) = \frac{96}{\left(1+4x\right)^4};~~~~
g^{\left(3\right)(x)} = -\frac{1536}{\left(1+4x\right)^5};~~~~
g^{\left(4\right)}(x) =\frac{30720}{\left(1+4x\right)^6}$$

$\star$~\textit{Công thức hình thang}\par
  
Ta có bảng giá trị:
\begin{longtable}{|c|c|Sc|Sc|}\hline
& $x_i$ & \multicolumn{2}{Sc|}{$y_i=g(x_i)= \frac{1}{(1+4x_i)^2}$}\\ \hline
\endhead
$0$ & $0,1$ & $\frac{25}{49}$ & \\ \hline
$1$ & $0,2$ & & $\frac{25}{81}$ \\ \hline
$2$ & $0.3$ & & $\frac{25}{121}$\\ \hline
$3$ & $0,4$ & & $\frac{25}{169}$\\ \hline
$4$ & $0,5$ & & $\frac{1}{9}$\\ \hline
$5$ & $0,6$ & & $\frac{25}{289}$ \\ \hline
$6$ & $0,7$ & & $\frac{25}{361}$ \\ \hline
$7$ & $0,8$ & & $\frac{25}{441}$\\ \hline
$8$ & $0,9$ & & $\frac{25}{529}$\\ \hline
$9$ & $1,0$ & & $\frac{1}{25}$\\ \hline
$10$ & $1,1$ & $\frac{25}{729}$ & \\ \hline
& & $0,5444976344$ & $1,03399924$ \\ \hline
\end{longtable}

Vậy theo công thức hình thang, giá trị gần đúng của tích phân cần tìm là:\par
$$I_T=\int\limits_{0,1}^{1,1} g(x)\mathrm{d}x \approx \frac{0,1}{2}\left[0,5444976344 + 2\left(1,03399924\right) \right] =0,1306248057$$

Nếu làm tròn đến năm chữ số thập phân thì $\overline{I}=0,13062$.\par

*~Đánh giá sai số theo công thức tích phân:\par

Ta có $M=\max_{0,1\leqslant x \leqslant 1,1}\left|g''(x) \right|= \left|g'' \left(1,1\right) \right|= 0,1129005854$.

Nên $\left|I-I_T\right|\leqslant \frac{M}{12}\left(1,1-0,1\right)\left(0,1\right)^2=9,408382116.10^{-5}$.\\
và $\left|I_T-\overline{I}\right| =4,8075.10^{-6}$.

Do đó $\left|I-\overline{I}\right| \leqslant \left| I- I_T \right| + \left| I_T -\overline{I} \right| \leqslant 9,408382116.10^{-5} + 4,8075.10^{-6}$.
